% This is LLNCS.DEM the demonstration file of
% the LaTeX macro package from Springer-Verlag
% for Lecture Notes in Computer Science,
% version 2.4 for LaTeX2e as of 16. April 2010
%
\documentclass[preprint]{sigplanconf}
%
\usepackage{url}
\usepackage{comment}
\usepackage{mathpartir}
\usepackage{listings}
\usepackage{alltt}
\usepackage{graphicx}
\usepackage{caption}
\usepackage{subfigure}
\usepackage{amssymb}
\usepackage{amsmath}
\usepackage{paralist, tabularx}
\usepackage{flushend}
\usepackage{multirow}
\usepackage{paralist}
\usepackage{amsthm}
\usepackage{algorithm}
\usepackage{algpseudocode}


\usepackage[flushleft]{threeparttable}
\usepackage{footnote}

\makeatletter
\def\BState{\State\hskip-\ALG@thistlm}
\makeatother

\def\UrlBreaks{\do\/\do-}

\usepackage{color}
\definecolor{egmcolor}{rgb}{1.0,0.0,0.722}
\newcommand*{\egm}[1]%
%%
%% a)
{\textcolor{egmcolor}{\noindent\textbf{[egm:~}\textit{#1}]}}
%%
% Other things...
\newcommand{\figref}[1]{Figure~\ref{#1}}
\newcommand{\defref}[1]{Definition~\ref{#1}}
\newcommand{\tableref}[1]{Table~\ref{#1}}
\newcommand{\secref}[1]{Section~\ref{#1}}
\newcommand{\lemmaref}[1]{Lemma~\ref{#1}}
\newcommand{\thmref}[1]{Theorem~\ref{#1}}
\newcommand{\algoref}[1]{Algorithm~\ref{#1}}

\newtheorem{definition}{Definition}
\newtheorem{theorem}{Theorem}
\newtheorem{lemma}{Lemma}

%
\begin{document}

\special{papersize=8.5in,11in}
\setlength{\pdfpageheight}{\paperheight}
\setlength{\pdfpagewidth}{\paperwidth}

\conferenceinfo{CONF 'yy}{Month d--d, 20yy, City, ST, Country} 
\copyrightyear{20yy} 
\copyrightdata{978-1-nnnn-nnnn-n/yy/mm} 
\doi{nnnnnnn.nnnnnnn}

\title{Static Analysis for Deadlock in MPI Programs}


\authorinfo{Yu Huang \and Eric Mercer}
           {Department of Computer Science\\
           Beigham Young University\\
           Provo, UT, 84602, USA}
           {\{yuHuang,egm\}@byu.edu}

\maketitle
%
%
\emergencystretch=1em


\begin{abstract}
The message passing interface (MPI) is a common programming model for distributed computing. Message race (intended or not) leads to non-determinism making test and debug in MPI programs very difficult. This paper addresses the NP--complete problem of deadlock detection in single-path MPI programs (e.g., programs with no branching). The solution uses progressively more precise analyses to generate and then prune a potential set of deadlocks: static matching to identify deadlock pattern instances; execution of the instances on an abstract machine to remove some that are non-feasible; and finally, if needed, validation of the instances to remove any remaining that are non-feasible. Novel in the work is the abstract machine based on counting to effectively reject non-feasible instances without exhaustively enumerating all message races. This paper further defines two distinct deadlock patterns with their validation: circular message dependency and orphaned receives. The paper includes a proof of each pattern showing them to be sound and complete in the approach. The efficiency of the approach is demonstrated in a set of benchmarks by comparing performance with two other solutions. The comparison shows an order of magnitude reduction in time and the ability to scale to millions of possible send-receive matches in the messaging.


\keywords
MPI, Static Analysis, Message Passing

\end{abstract}

%% Commands For the syntax EBNF
\newcommand{\cfgnt}[1]{\emph{#1}}
\newcommand{\cfgq}[1]{\texttt{#1}}
\newcommand{\cfgt}[1]{\textbf{#1}}
\newcommand{\cfglhs}[1]{\cfgnt{#1} & $::=$}
\newcommand{\cfgrule}[2]{\cfglhs{#1} & #2 \\}
\newcommand{\cfgor}{\textbar\ }
\newcommand{\cfgstart}{\begin{tabular}{r@{\hspace{1mm}}r@{\hspace{2mm}}l}}
\newcommand{\cfgend}{\end{tabular}}
\newcommand{\cfgline}[1]{ ~ && #1 \\ }
\newcommand{\cfglinetab}[1]{ ~ && \hspace{1cm} #1 \\ }
\newcommand{\cfgorline}[1]{ ~ & \cfgor & #1 \\ }
\newcommand{\lp}{\cfgq{(}}
\newcommand{\rp}{\cfgq{)}}
\newcommand{\lb}{\cfgq{[}}
\newcommand{\rb}{\cfgq{]}}

% ---------------------------------------------------------------------
% Save boxes for the various figures in the example section
% ---------------------------------------------------------------------
\newcommand{\epsnd}{\textit{$N_s$}}
\newcommand{\eprcv}{\textit{$N_r$}}
\newcommand{\epwait}{\textit{$P_r$}}
\newcommand{\epbarrier}{\textit{$N_b$}}
\newcommand{\frm}{\textit{frm}}
\newcommand{\vbot}{\textit{v$-\bot$}}
\newcommand{\status}{\textit{s}}
\newcommand{\error}{\ensuremath{\mathbf{error}}}
\newcommand{\aidmap}{\textit{A}}
\newcommand{\thread}{\ensuremath{\mathit{p}}}
\newcommand{\aid}{\ensuremath{\mathit{x}}}
\newcommand{\num}{\ensuremath{\mathit{v}}}
\newcommand{\rcvp}{\ensuremath{\mathit{pt}}}
\newcommand{\npro}{\ensuremath{\mathit{n}}}
\newcommand{\cmd}{\ensuremath{\mathit{e}}}
\newcommand{\op}{\ensuremath{\mathit{op}}}
\newcommand{\comm}{\ensuremath{\mathit{cm}}}
\newcommand{\applyop}{\ensuremath{\mathrm{op}}}
\newcommand{\wait}{\ensuremath{\mathbf{w}}}
\newcommand{\sendi}{\ensuremath{\mathbf{s}}}
\newcommand{\recvi}{\ensuremath{\mathbf{r}}}
\newcommand{\barrier}{\ensuremath{\mathbf{b}}}
\newcommand{\traceentry}{\ensuremath{\sigma}}
\newcommand{\movelist}{\ensuremath{\delta}}
\newcommand{\ep}{\ensuremath{\mathbf{\gamma}}}
\newcommand{\src}{\ensuremath{\alpha}}
\newcommand{\dst}{\ensuremath{\beta}}
\newcommand{\true}{\ensuremath{\mathbf{true}}}
\newcommand{\false}{\ensuremath{\mathbf{false}}}
\newcommand{\match}{\ensuremath{\mathrm{m}}}
\newcommand{\matchl}{\ensuremath{\mathrm{match}}}
\newcommand{\findrecv}{\ensuremath{\mathrm{search_r}}}
\newcommand{\smt}{\ensuremath{\mathit{smt}}}

\newcommand{\statuschange}{\left\{ \begin{array}{ll}  \status &\ \mathrm{if}\ |\epsnd(\dst)(\src)|>0\\
   \mathit{error} &\  \mathrm{otherwise}\end{array}\right .}

\newcommand{\reduce}[1]{\ensuremath{\rightarrow_{#1}}}
% Multiple reductions
\newcommand{\reduceK}[1]{\ensuremath{\rightarrow_{#1}^{*}}}
% Non-deterministic reductions
\newcommand{\reduceN}[1]{\ensuremath{\dashrightarrow_{#1}}}
\newcommand{\reduceNK}[1]{\ensuremath{\dashrightarrow_{#1}^{*}}}
\newcommand{\mt}{\ensuremath{mt}}
\newcommand{\trace}{\ensuremath{\mathit{trace}}}
\newcommand{\movebot}{\ensuremath{\mathit{m}}}
\newcommand{\ret}{\ensuremath{\mathbf{ret}}}


\newsavebox{\boxTZero}
\begin{lrbox}{\boxTZero}
\begin{minipage}[t]{0.65\linewidth}
\large
\begin{alltt}
00 (\(\mathbf{s}\) \(\mathit{s\sb{0}}\) \(\mathit{0 1}\))
01 (\(\mathbf{r}\) \(\mathit{r\sb{0}}\) \(\mathit{\ast 0}\) \(\mathit{w\sb{0}}\))
02 (\(\mathbf{w}\) \(\mathit{w\sb{0}}\)) 
03 \underline{(\(\mathbf{r}\) \(\mathit{r\sb{1}}\) \(\mathit{1 0}\) \(\mathit{w\sb{1}}\))}
04 (\(\mathbf{w}\) \(\mathit{w\sb{1}}\))
\end{alltt}
\end{minipage}
\end{lrbox}

\newsavebox{\boxTOne}
\begin{lrbox}{\boxTOne}
\begin{minipage}[t]{0.65\linewidth}
\large
\begin{alltt}
10 (\(\mathbf{r}\) \(\mathit{r\sb{2}}\) \(\mathit{\ast 1}\) \(\mathit{w\sb{2}}\))
11 (\(\mathbf{w}\) \(\mathit{w\sb{2}}\))
12 (\(\mathbf{s}\) \(\mathit{s\sb{1}}\) \(\mathit{1 0}\))
\end{alltt}
\end{minipage}
\end{lrbox}

\newsavebox{\boxTTwo}
\begin{lrbox}{\boxTTwo}
\begin{minipage}[t]{0.65\linewidth}
\large
\begin{alltt}
20 (\(\mathbf{s}\) \(\mathit{s\sb{2}}\) \(\mathit{2 0}\))
\end{alltt}
\end{minipage}
\end{lrbox}

% ---------------------------------------------------------------------
% END Save boxes
% ---------------------------------------------------------------------

\newcommand\examplefigone{
\begin{figure*}[tb]
\begin{center}
\setlength{\tabcolsep}{2pt}
\begin{tabular}[t]{c|c|c}
$\mathit{p_0}$ & $\mathit{p_1}$ & $\mathit{p_2}$ \\
\hline
\scalebox{0.8}{\usebox{\boxTZero}}&
\scalebox{0.8}{\usebox{\boxTOne}} &
\scalebox{0.8}{\usebox{\boxTTwo}}\\
\end{tabular}
\end{center}
\caption{A Deadlock Caused by Orphaned Receive}
\label{fig:deadlock1}
\end{figure*}
}



\newsavebox{\boxnozero}
\begin{lrbox}{\boxnozero}
\begin{minipage}[t]{0.65\linewidth}
\large
\begin{alltt}	
00 (\(\mathbf{r}\) \(\mathit{r\sb{0}}\) \(\mathit{\ast 0}\) \(\mathit{w\sb{0}}\))
01 (\(\mathbf{w}\) \(\mathit{w\sb{0}}\))
02 (\(\mathbf{s}\) \(\mathit{s\sb{0}}\) \(\mathit{0 1}\)) 
03 \underline{(\(\mathbf{r}\) \(\mathit{r\sb{1}}\) \(\mathit{1 0}\) \(\mathit{w\sb{1}}\))}
04 (\(\mathbf{w}\) \(\mathit{w\sb{1}}\)) 
\end{alltt}
\end{minipage}
\end{lrbox}

\newsavebox{\boxnoone}
\begin{lrbox}{\boxnoone}
\begin{minipage}[t]{0.65\linewidth}
\large
\begin{alltt}
10 (\(\mathbf{r}\) \(\mathit{r\sb{2}}\) \(\mathit{\ast 1}\) \(\mathit{w\sb{2}}\)) 
11 (\(\mathbf{w}\) \(\mathit{w\sb{2}}\))
12 (\(\mathbf{s}\) \(\mathit{s\sb{1}}\) \(\mathit{1 0}\))
\end{alltt}
\end{minipage}
\end{lrbox}

\newsavebox{\boxnotwo}
\begin{lrbox}{\boxnotwo}
\begin{minipage}[t]{0.65\linewidth}
\large
\begin{alltt}
20 (\(\mathbf{s}\) \(\mathit{s\sb{2}}\) \(\mathit{2 0}\))
\end{alltt}
\end{minipage}
\end{lrbox}


\newcommand\examplefigtwo{
\begin{figure*}[tb]
\begin{center}
\setlength{\tabcolsep}{2pt}
\begin{tabular}[t]{c|c|c}
$\mathit{p_0}$ & $\mathit{p_1}$ & $\mathit{p_2}$ \\
\hline
\scalebox{0.8}{\usebox{\boxnozero}}&
\scalebox{0.8}{\usebox{\boxnoone}} &
\scalebox{0.8}{\usebox{\boxnotwo}}\\
\end{tabular}
\end{center}
\caption{No Deadlock Caused by Orphaned Receive}
\label{fig:nodeadlock1}
\end{figure*}
}


\newsavebox{\boxone}
\begin{lrbox}{\boxone}
\begin{minipage}[t]{0.65\linewidth}
\large
\begin{alltt}
00 (\(\mathbf{s}\) \(\mathit{s\sb{0}}\) \(\mathit{0 1}\)) 
01 \underline{(\(\mathbf{r}\) \(\mathit{r\sb{0}}\) \(\mathit{\ast 0}\) \(\mathit{w\sb{0}}\))}
02 (\(\mathbf{w}\) \(\mathit{w\sb{0}}\)) 
03 (\(\mathbf{s}\) \(\mathit{s\sb{1}}\) \(\mathit{0 1}\))
\end{alltt}
\end{minipage}
\end{lrbox}

\newsavebox{\boxtwo}
\begin{lrbox}{\boxtwo}
\begin{minipage}[t]{0.65\linewidth}
\large
\begin{alltt}
10 (\(\mathbf{r}\) \(\mathit{r\sb{1}}\) \(\mathit{\ast 1}\) \(\mathit{w\sb{1}}\)) 
11 (\(\mathbf{w}\) \(\mathit{w\sb{1}}\))
12 \underline{(\(\mathbf{r}\) \(\mathit{r\sb{2}}\) \(\mathit{\ast 1}\) \(\mathit{w\sb{2}}\))}
13 (\(\mathbf{w}\) \(\mathit{w\sb{2}}\)) 
14 (\(\mathbf{s}\) \(\mathit{s\sb{2}}\) \(\mathit{1 2}\))
\end{alltt}
\end{minipage}
\end{lrbox}

\newsavebox{\boxthree}
\begin{lrbox}{\boxthree}
\begin{minipage}[t]{0.65\linewidth}
\large
\begin{alltt}
20 \underline{(\(\mathbf{r}\) \(\mathit{r\sb{3}}\) \(\mathit{1 2}\) \(\mathit{w\sb{3}}\))}
21 (\(\mathbf{w}\) \(\mathit{w\sb{3}}\))
22 (\(\mathbf{s}\) \(\mathit{s\sb{3}}\) \(\mathit{2 0}\))
\end{alltt}
\end{minipage}
\end{lrbox}

% ---------------------------------------------------------------------
% END Save boxes
% ---------------------------------------------------------------------

\newcommand\examplefigthree{
\begin{figure*}[tb]
\begin{center}
\setlength{\tabcolsep}{2pt}
\begin{tabular}[t]{c|c|c}
$p_0$ & $p_1$ & $p_2$ \\
\hline
\scalebox{0.8}{\usebox{\boxone}}&
\scalebox{0.8}{\usebox{\boxtwo}} &
\scalebox{0.8}{\usebox{\boxthree}}\\
\end{tabular}
\end{center}
\caption{A Deadlock Caused by Circular Dependency in Messages}
\label{fig:deadlock2}
\end{figure*}
}


\newsavebox{\boxoneno}
\begin{lrbox}{\boxoneno}
\begin{minipage}[t]{0.65\linewidth}
\large
\begin{alltt}
00 (\(\mathbf{s}\) \(\mathit{s\sb{0}}\) \(\mathit{0 1}\))
01 \underline{(\(\mathbf{r}\) \(\mathit{r\sb{0}}\) \(\mathit{\ast 0}\) \(\mathit{w\sb{0}}\))}
02 (\(\mathbf{w}\) \(\mathit{w\sb{0}}\))
03 (\(\mathbf{s}\) \(\mathit{s\sb{1}}\) \(\mathit{0 1}\))
\end{alltt}
\end{minipage}
\end{lrbox}

\newsavebox{\boxtwono}
\begin{lrbox}{\boxtwono}
\begin{minipage}[t]{0.65\linewidth}
\large
\begin{alltt}
10 \underline{(\(\mathbf{r}\) \(\mathit{r\sb{1}}\) \(\mathit{\ast 1}\) \(\mathit{w\sb{1}}\))}
11 (\(\mathbf{w}\) \(\mathit{w\sb{1}}\))
12 (\(\mathbf{s}\) \(\mathit{s\sb{2}}\) \(\mathit{1 2}\))
\end{alltt}
\end{minipage}
\end{lrbox}

\newsavebox{\boxthreeno}
\begin{lrbox}{\boxthreeno}
\begin{minipage}[t]{0.65\linewidth}
\large
\begin{alltt}
20 \underline{(\(\mathbf{r}\) \(\mathit{r\sb{2}}\) \(\mathit{1 2}\) \(\mathit{w\sb{2}}\))}
21 (\(\mathbf{w}\) \(\mathit{w\sb{2}}\))
22 (\(\mathbf{s}\) \(\mathit{s\sb{3}}\) \(\mathit{2 0}\))
\end{alltt}
\end{minipage}
\end{lrbox}

% ---------------------------------------------------------------------
% END Save boxes
% ---------------------------------------------------------------------

\newcommand\examplefigfour{
\begin{figure*}[tb]
\begin{center}
\setlength{\tabcolsep}{2pt}
\begin{tabular}[t]{c|c|c}
$p_0$ & $p_1$ & $p_2$ \\
\hline
\scalebox{0.8}{\usebox{\boxoneno}}&
\scalebox{0.8}{\usebox{\boxtwono}} &
\scalebox{0.8}{\usebox{\boxthreeno}}\\
\end{tabular}
\end{center}
\caption{No Deadlock Caused by Circular Dependency in Messages}
\label{fig:nodeadlock2}
\end{figure*}
}

\newsavebox{\boxonezerobuffer}
\begin{lrbox}{\boxonezerobuffer}
\begin{minipage}[t]{0.65\linewidth}
\large
\begin{alltt}
00 (\(\mathbf{r}\) \(\mathit{r\sb{0}}\) \(\mathit{\ast 0}\) \(\mathit{w\sb{0}}\)) 
01 (\(\mathbf{w}\) \(\mathit{w\sb{0}}\) \(\mathit{r\sb{0}}\))
02 (\(\mathbf{s}\) \(\mathit{0 1}\)) 
03 (\(\mathbf{r}\) \(\mathit{r\sb{1}}\) \(\mathit{\ast 0}\) \(\mathit{w\sb{1}}\)) 
04 (\(\mathbf{w}\) \(\mathit{w\sb{1}}\) \(\mathit{r\sb{1}}\))
\end{alltt}
\end{minipage}
\end{lrbox}

\newsavebox{\boxtwozerobuffer}
\begin{lrbox}{\boxtwozerobuffer}
\begin{minipage}[t]{0.65\linewidth}
\large
\begin{alltt}
10 (\(\mathbf{s}\) \(\mathit{1 0}\)) 
11 (\(\mathbf{r}\) \(\mathit{r\sb{2}}\) \(\mathit{\ast 1}\) \(\mathit{w\sb{2}}\)) 
12 (\(\mathbf{w}\) \(\mathit{w\sb{2}}\) \(\mathit{r\sb{2}}\))
\end{alltt}
\end{minipage}
\end{lrbox}

\newsavebox{\boxthreezerobuffer}
\begin{lrbox}{\boxthreezerobuffer}
\begin{minipage}[t]{0.65\linewidth}
\large
\begin{alltt}
20 (\(\mathbf{s}\) \(\mathit{2 0}\)) 
\end{alltt}
\end{minipage}
\end{lrbox}


\newcommand\examplefigfive{
%:
\begin{figure*}[tb]
\begin{center}
\setlength{\tabcolsep}{2pt}
\begin{tabular}[t]{c|c|c}
$p_0$ & $p_1$ & $p_2$ \\
\hline
\scalebox{0.8}{\usebox{\boxonezerobuffer}}&
\scalebox{0.8}{\usebox{\boxtwozerobuffer}} &
\scalebox{0.8}{\usebox{\boxthreezerobuffer}}\\
\end{tabular}
\end{center}
\caption{A Deadlock for Zero Buffer Semantics}
\label{fig:zeropattern}
\end{figure*}
}



\section{Introduction}
Message Passing Interface (MPI) is widely used in high performance computing (HPC). A common problem in any MPI program, is communication deadlock. A communication deadlock is ``\emph{a situation in which each member process of the group is waiting for some member process to communicate with it, but no member is attempting to communicate with it}" \cite{DBLP:conf/fsttcs/Natarajan84}. This paper refers to a communication deadlock as deadlock in the following discussion. 

This paper considers only single-path MPI programs because: 1) they are typical of many HPC applications and the standard for most deadlock analysis tools, and 2) the deadlock problem for a single-path MPI program is NP--Complete \cite{DBLP:conf/fm/ForejtKNS14}. There are two reasons that lead deadlocks in a single-path message passing program hard to detect. In particular, messages are communicated in a non-deterministic way such that a receive may be matched with more than one send in the runtime system. Also, the message communication is impacted by two buffering semantics: infinite buffer semantics (messages are buffered in the system) and zero buffer semantics (no buffering in the system). 

Several solutions are proposed for deadlock detection. Dynamic analysis includes the POE algorithm that is capable of analyzing the behavior of Message Passing Interface (MPI) programs \cite{DBLP:conf/ppopp/VakkalankaSGK08}. This algorithm is implemented by a modern MPI verifier, ISP. An extension is the algorithm MSPOE that is designed to detect deadlock in an MPI program \cite{DBLP:conf/sbmf/SharmaGB12}. The problem can also be checked by SAT/SMT technique. Forejt et al. proposed a SAT based approach to detect deadlock in a single-path MPI program \cite{DBLP:conf/fm/ForejtKNS14}. As far as we know, these two approaches are the only existing works in message passing deadlock detection. Unfortunately, they do not scale well for several benchmark program. This paper show the experimental result later. In the field of shared memory programs, Joshi et al. proposed a technique that uses a random thread scheduler to create the potential deadlocks that are detected by an imprecise dynamic analyzer \cite{DBLP:conf/pldi/JoshiPSN09}. This approach is able to check large multi-threaded programs and does not give any false warnings. 

This paper presents a new algorithm that is able to detect a deadlock for a single-path MPI program in three steps. First, the algorithm uses a static analyzer to detect a set of pattern instances by statically traversing the program. The algorithm then uses an abstract machine to prune non-feasible pattern instances from the set of potential deadlocks. If needed, the algorithm finally validates whether or not any feasible instance implies a real deadlock. Instead of exhaustively enumerating all message races, the novelty is that the abstract machine efficiently rejects non-feasible instances by simply counting the issued sends and receives. The complexity of the algorithm is quadratic. 

The paper further defines two distinct patterns of deadlock and their validations: circular dependency and orphaned receive. A circular dependency pattern may cause a program deadlock if there exists a cycle among a group of processes where a receive on each member process waits the issuing of a send on another member process but never gets a response. An orphaned receive pattern may also cause a program deadlock if there exists a pair of a wildcard receive (a receive that may match a send from any source) and a deterministic receive (a receive that may match a send from a fixed source) on an identical process. The deadlock occurs when the wildcard receive is matched with a send that should match the deterministic receive in runtime. The paper also presents how to detect and validate the instances for each pattern above.

The paper additionally gives proofs that the algorithm is sound and complete for circular dependency pattern and that it is sound for orphaned receive pattern. To make the algorithm complete for orphaned receive pattern, a high-cost SMT encoding extended from the prior work (\cite{DBLP:conf/kbse/HuangMM13,HuangNFM15}) is required to identify a real deadlock. As a note, the algorithm is designed for infinite buffer semantics. Nevertheless, it is adaptable to zero buffer semantics with a few changes in deadlock validation.

The contributions are summarized as follows.
\begin{itemize}
\item The key contribution is an abstract machine that is able to statically prune non-feasible pattern instances from a set of potential deadlocks without false warnings; 
\item The second contribution is the definition of two typical patterns: circular dependency and orphaned receive, and their validations;
\item The third contribution is the implementation of the algorithm in the paper for MPI deadlock detection and a set of benchmarks that demonstrate it is more efficient than two state-of-art MPI verifiers.
\end{itemize}

The rest of the paper is organized as follows: Sections 2 introduces a few notations and definitions for an MPI program; Section 3 presents the main framework of the technique in the paper with a soundness and completeness proof, Section 4 and 5 present the circular dependency pattern and the orphaned receive pattern with a few examples, the pattern matching algorithms and the validation algorithms; Section 6 gives the experimental results; Section 7 discusses the related work; and Section 8 discusses the conclusion and future work.

\section{Background}
This presentation uses a CTP syntax with semantics loosely based on MPI semantics for convenience. In an MPI program, it is assumed that only non-blocking sends and receives are used. A non-blocking operation is simply consumed in program execution.
%% introduce the language aspects: non-blocking, blocking, barrier, collective, point-to-point in the introduction.
In general, the syntax of an MPI program is presented in \figref{fig:expr:stx} (a). The syntax describes a concurrent trace program (CTP). The presentation uses ellipses ($\ldots$) to represent zero or more repetitions, bold-face to indicate terminals. The language defines a CTP (\textit{ctp}) as a list of processes.  A process (\thread) is a list of commands.  For simplicity, commands (\textit{e}) are restricted to non-blocking send (\sendi), non-blocking receive (\recvi), wait (\wait) and barrier (\barrier). Each command is a tuple. It is referred to each tuple as a single notation $\mathtt{o_c}$ in the following discussion, where $\mathtt{o}\in\{\sendi,\recvi,\wait,\barrier\}$ and $\mathtt{c}$ is the line number.  
The non-terminal \aid\ in the grammar is a unique string identifier \textbf{ID} associated with a receive or a wait. The wait takes a second identifier belonging to the associated receive. Similarly, a receive is paired with a wait by recording the identifier of this wait in the tuple. It is assumed that a message is sent out immediately once the non-blocking send is issued, therefore, it is not necessary to associate an identifier or a wait with the send. A barrier \barrier\ is also associated with an identifier that is unique for its communicator. A communicator identifies a group of barriers. The non-terminal \num\ in the grammar is a number that represents an endpoint. To be precise, a receive (or a send) is associated with a source endpoint and a destination endpoint. Note that if the source endpoint of a receive is $\ast$, this receive is a wildcard receive meaning it may be matched with a send from any source. The set \rcvp\ is a deadlock pattern instance that consists of a set of receives. The patterns are defined in the following section. The data and computation are ignored since only message communication is considered in the problem of deadlock. 

\newsavebox{\boxLangSyntax}

\begin{lrbox}{\boxLangSyntax}
\begin{minipage}[c]{0.3\linewidth}
\cfgstart
\cfgrule{ctp}{\lp\cfgnt{\thread}$~\ldots$\rp}
\cfgrule{\thread}{\lp\cfgnt{\cmd}~$\ldots$~$\bot$\rp}
\cfgrule{\cmd}{\lp\cfgt{\sendi}~\cfgt{\num}~\cfgt{\num}\rp}
   \cfgorline{\lp\cfgt{\recvi}~\cfgt{\aid}~\cfgt{\num}~\cfgt{\num}~\cfgt{\aid}\rp}
   \cfgorline{\lp\cfgt{\wait}~\cfgt{\aid}~\cfgt{\aid}\rp}
   \cfgorline{\lp\cfgt{\barrier}~\cfgt{\aid}\rp}
\cfgrule{\num}{\cfgt{number}}
\cfgrule{\rcvp}{\lp\aid~$\ldots$\rp}
\cfgrule{\npro}{\cfgt{number}}
\cfgrule{\aid}{\cfgt{ID}}
\cfgend
\end{minipage}
\end{lrbox}


\newsavebox{\boxEvalSyntax}
\begin{lrbox}{\boxEvalSyntax}
\begin{minipage}[c]{0.3\linewidth}
\cfgstart
\cfgrule{st}{\lp\cfgnt{ctp}\ \cfgnt{ctp}\ \cfgnt{\epsnd}\ \cfgnt{\eprcv}\ \cfgnt{\epwait}\ \cfgnt{\epbarrier}\ \rcvp\rp}
\cfgrule{\epsnd}{\cfgt{\mt}\
   \cfgor\lp\cfgnt{\epsnd}~\lb\lp\cfgt{\num},\cfgt{\num}\rp~$\rightarrow$~\cfgt{\num}\rb\rp}
\cfgrule{\eprcv}{\cfgt{\mt}\
   \cfgor\lp\cfgnt{\eprcv}~\lb\lp\cfgt{\num},\cfgt{\num}\rp~$\rightarrow$~\cfgt{\num}\rb\rp}
\cfgrule{\epwait}{\cfgt{\mt}\
   \cfgor\lp\cfgnt{\epwait}~\lb\cfgnt{\aid}~$\rightarrow$~\cfgnt{rcv}\rb\rp}
\cfgrule{rcv}{\lp\lb\cfgt{\aid}~\cfgt{\num}~\cfgt{\num}\rb\ \ldots\rp}
\cfgrule{\epbarrier}{\cfgt{\mt}\
   \cfgor\lp\cfgnt{\epbarrier}~\lb\cfgnt{\aid}~$\rightarrow$~\cfgt{\num}\rb\rp}
\cfgend
\end{minipage}
\end{lrbox}

\begin{figure}
\begin{center}
\setlength{\tabcolsep}{15pt}
\begin{tabular}{cc}
\scalebox{0.75}{\usebox{\boxLangSyntax}}
&
\scalebox{0.75}{\usebox{\boxEvalSyntax}}
\\ \\
(a) & (b)
\end{tabular}
\end{center}
\caption{The language syntax with its evaluation syntax for the operational semantics--bold face indicates a terminal. (a) The input syntax. (b) The evaluation syntax.}
\label{fig:expr:stx}
\end{figure}

%%TODO: may move the machine state to where presenting the abstract machine.



\figref{fig:deadlock2} is an example CTP with line numbers and underlined commands to indicate a pattern instance. Process $p_0$ sends a message to $p_1$, then receives a message from any source, and finally sends another message to $p_1$. Process $p_1$ receives two messages from any source in the receives $r_1$ and $r_2$, and then sends a message to $p_2$. Process $p_2$ receives a message from $p_1$ and then sends a message to $p_0$. The tuple $(r_0,r_2,r_3)$ is a pattern instance. If each process is able to arrive at these program points in some feasible execution, then the program contains the deadlock indicated by the pattern instance, which in this example, is a circular dependency.

\section{Main Framework}

%%The general algorithm including three steps: Pattern Match, Feasible Check and Validation. The pseudocode needs to be revised so it shows each step. 


\algoref{algo:main} describes the general structure of the approach in this paper which consists of three distinct steps: pattern matching (line 1), feasibility checking (line 3), and validating (line 6). $\mathrm{PATTERNMATCH}$ statically generates a set ($\mathit{PT}$) of matched pattern instances in the \emph{ctp} with the help of an additional input, $M$, that defines all the potential match-pairs in the program. $M$ can be generated in quadratic time \cite{DBLP:conf/kbse/HuangMM13}. 

$\mathrm{FEASIBLECHECK}$ is an abstract machine to prune pattern instances for which it is possible prove no feasible schedule exists. In other words, it is provably not possible to execute the \emph{ctp} such that each process associated with a receive ID in the pattern instance is at that receive. The machine removes matched receive IDs from the pattern instance as it executes the \emph{ctp}, so if $\mathit{pt^\prime}$ is not empty upon return, the pattern instance is provably not feasible and the algorithm continues with the next pattern instance (line 5). The algorithm additionally returns the number of issued sends ($\mathit{N_s}$) and the number of issued receives ($\mathit{N_r}$). These are used in validation depending on the type of the pattern instance: circular dependency or orphaned receive.

%The framework is useful for error detection with appropriate pattern match and validation algorithms. This paper applies the framework for deadlock detection based on two types of deadlock patterns in the next two sections.

\begin{algorithm}[t]
\caption{Main Framework}\label{algo:main}
\begin{algorithmic}[1]
%\Procedure{Main Entrance}{}
%\Require $\mathit{ctp}$, a single-path MPI program
%\Require $\mathit{M}(\mathtt{r_c}) = \{\mathtt{s_l}\mid(\mathtt{s_l},\mathtt{r_c})\in\mathit{M}\}$, a set of potentially matched sends for $\mathtt{r_c}$
%\State  $\mathit{PT}$, a set of pattern instances
%\State  $\mathit{pt}$, a set of receives in the pattern instance $\mathit{pt}\in\mathit{PT}$
\State  $\mathit{PT} \gets$ \Call {PatternMatch}{$\mathit{ctp}$, $\mathit{M}$}
\For{$\mathit{pt} \in \mathit{PT}$}
\State ($\mathit{N_s}, \mathit{N_r}, \mathit{pt^\prime})\gets$\Call {FeasibleCheck}{$\mathit{pt}$, $\mathit{ctp}$}
\If{$\mathit{pt^\prime} \neq \emptyset$}
\State continue.
\ElsIf{$\neg$\Call {Validate}{$\mathit{N_s}, \mathit{N_r}, \mathit{pt}$}}
\State continue.
\Else\ report error and exit.
\EndIf

%\State ($\mathit{ctp}_s, \mathit{N_s}, \mathit{N_r}, \mathit{empty}_{pt}) \gets$ \Call {ScheduleFinder}{$\mathit{pt}$}
%\If{$\mathit{empty}_{pt}$} 
%\If{\Call{isCircular}{$\mathit{pt}$}}
%\For{$\mathtt{r_c}\in\mathit{pt}$}
%\State $\mathit{src} \gets$ source endpoint of $\mathtt{r_c}$, $\mathit{dest} \gets$ destination endpoint of $\mathtt{r_c}$
%\If{$\mathit{N_s}(\mathit{dest},\mathit{src}) > \mathit{N_r}(\mathit{dest},\mathit{src})$}
%\State \textbf{continue} \textit{point}
%\EndIf
%\EndFor
%\State report deadlock and exit.
%\EndIf
%\If{\Call {isMismatch}{$\mathit{pt}} \in$}
%\If{\textproc{SAT}({\Call {Encode}{$\mathit{ctp}_s$}})}
%\State report deadlock and exit.
%\EndIf
%\EndIf
%\EndIf
\EndFor
%\EndProcedure
\end{algorithmic}
\end{algorithm}

%%The feasible check and the soundness proof of feasible check (assuming pattern match is sound and validation is sound and complete)

$\mathrm{VALIDATE}$ proves a pattern instance feasible, which means that it is a real deadlock in the \emph{ctp}. If the deadlock is real, the algorithm reports the error and exits (line 8); otherwise it continues with the next pattern instance.
\textrm{FEASIBLECHECK} is able to efficiently prune schedules that are provably non-feasible with predictive analysis using counting and an auxiliary data structures to track FIFO ordering on messages. 

\figref{fig:machine} is a term rewriting system for a syntactic
machine (i.e., the machine state is represented by a string) for
$\mathrm{FEASIBLECHECK}$ to prove that a pattern instance is not
feasible. \figref{fig:expr:stx}(b) is the syntax for that machine. The
rewrites define how the machine executes an input \emph{ctp} and
pattern instance \emph{pt} by evolving the machine state. At a high-level, the machine
\begin{compactitem}
\item Process a send, receive, or barrier by counting the send (\emph{Sndi Command}), queuing up the receive on the indicated wait ID (\emph{Rcvi Command}), or counting the barrier (\emph{Barrier Command 1}).
\item Consume a waits when the associated queue is empty (\emph{Wait (Rcvi) Command 1})
\item Remove a receive ID from the pattern instance when the associated receive is next on the queue for the indicated wait (\emph{Wait (Rcvi) Command 2}).
\item Remove the next receive from the queue for the indicated wait and update the number of receives when feasible (\emph{Wait (Rcvi) Command 3}).
\item Consume a barrier if all the processes have arrived (\emph{Barrier Command 2})---communicator groups may only be used once.
\end{compactitem}
At a lower-level, the machine state (\textit{st}) is a six-tuple of variables. The first variable \textit{ctp} defines the concurrent trace program being analyzed. The set \epsnd\ maps a destination endpoint and a source endpoint to a number that is used to count issued sends. The variable \eprcv\ has the same structure only the number is used to count the number of matched receives. The variable \epwait\ records the pending receives by mapping the unique identifier of a wait to a set of the issued receives $\mathit{rcv}$. 
%A nearest-enclosing wait is the first wait that witnesses the completion of a receive by indicating that the message is delivered and that all the previous receives on the same process issued earlier are complete as well. 
In this set, the action identifier, the source endpoint and the destination endpoint are recorded for each receive. The variable \epbarrier\ maps the unique identifier of a communicator to a number that is used to count the number of witnessed barriers.

\begin{figure*}[tb]
\centering
\scalebox{0.9}{
\mprset{flushleft}
\begin{mathpar}

\inferrule[Sndi Command]{
  \epsnd(v_{to},v_{frm}) = v_c \\ \epsnd^\prime = \epsnd[(v_{to},v_{frm}) \mapsto v_c +1] \\ 
  %\epsnd^{\prime\prime} = \epsnd^\prime[\forall v\ldotp \epsnd(v_{to},v) = v_i \mid (v_{to},v) \mapsto v_i +1] 
 \epsnd(v_{to},\ast) = v_i \\ \epsnd^{\prime\prime} = \epsnd^\prime[(v_{to},\ast) \mapsto v_i +1] 
}{
  ((\thread_0\ \ldots\ ((\sendi\ \aid\ v_{frm}\ v_{to})\ \cmd_1\ \ldots\ \bot)\ \thread_2\ \ldots)\ \epsnd\ \eprcv\ \epwait\ \epbarrier\ \rcvp) \reduce{m}
  ((\thread_0\ \ldots\ (\cmd_1\ \ldots\ \bot)\ \thread_2\ \ldots)\ \epsnd^{\prime\prime}\ \eprcv\ \epwait\ \epbarrier\ \rcvp)
}

\and

\inferrule[Rcvi Command]{
 \epwait(\aid_w) =  ([\aid_1\ v_{frm1}\ v_{to1}]\ \ldots)
 \\ \epwait^\prime = \epwait [ \aid_w \mapsto ([\aid_0\ v_{frm0}\ v_{to0}]\ [\aid_1\ v_{frm1}\ v_{to1}]\ \ldots])] 
}{
  ((\thread_0\ \ldots\ ((\recvi\ \aid_0\ v_{frm0}\ v_{to0}\ \aid_w)\ \cmd_1\ \ldots\ \bot)\ \thread_2\ \ldots)\ \epsnd\ \eprcv\ \epwait\ \epbarrier\ \rcvp) \reduce{m}
  ((\thread_0\ \ldots\ (\cmd_1\ \ldots\ \bot)\ \thread_2\ \ldots)\ \epsnd\ \eprcv\ \epwait^\prime\ \epbarrier\ \rcvp)
}
\and
\inferrule[Wait (rcvi) Command 1]
{
  \epwait(\aid_w) = ()
}{
  ((\thread_0\ \ldots\ ((\wait\ \aid_w)\ \cmd_1\ \ldots\ \bot)\ \thread_2\ \ldots)\ \epsnd\ \eprcv\ \epwait\ \epbarrier\ \rcvp) \reduce{m}
  ((\thread_0\ \ldots\ (\cmd_1\ \ldots\ \bot)\ \thread_2\ \ldots)\ \epsnd\ \eprcv\ \epwait\ \epbarrier\ \rcvp)
}
\and
\inferrule[Wait (rcvi) Command 2]
{
   \epwait(\aid_w) = ([\aid_0\ v_{frm0}\ v_{to0}]\ [\aid_1\ v_{frm1}\ v_{to1}]\ \ldots) \\ \rcvp = (\aid_a\ \ldots\ \aid_0\ \aid_b\ \ldots) \\ \aid_0 \in \rcvp \\ \rcvp^\prime = (\aid_a\ \ldots\ \aid_b\ \ldots)
}{
  ((\thread_0\ \ldots\ ((\wait\ \aid_w)\ \cmd_1\ \ldots\ \bot)\ \thread_2\ \ldots)\ \epsnd\ \eprcv\ \epwait\ \epbarrier\ \rcvp) \reduce{m}
  ((\thread_0\ \ldots\ ((\wait\ \aid_w)\ \cmd_1\ \ldots\ \bot)\ \thread_2\ \ldots)\ \epsnd\ \eprcv\ \epwait\ \epbarrier\ \rcvp^\prime)
}
\and
\inferrule[Wait (rcvi) Command 3]
{
  \epwait(\aid_w) = ([\aid_0\ v_{frm0}\ v_{to0}]\ [\aid_1\ v_{frm1}\ v_{to1}]\ \ldots) \\
  \aid_0 \notin \rcvp \\ \eprcv(v_{to0},v_{frm0}) < \epsnd(v_{to0},v_{frm0}) \\ \eprcv(v_{to0},\ast) < \epsnd(v_{to0},\ast) \\
   \eprcv(v_{to0},v_{frm0}) = v_c \\
    \eprcv^\prime = \eprcv [(v_{to0}, v_{frm0}) \mapsto v_c + 1]] \\ 
    \epwait^\prime = \epwait [\aid_w\ \mapsto\ ([\aid_1\ v_{frm1}\ v_{to1}]\ \ldots)]
}{
  ((\thread_0\ \ldots\ ((\wait\ \aid_w)\ \cmd_1\ \ldots\ \bot)\ \thread_2\ \ldots)\ \epsnd\ \eprcv\ \epwait\ \epbarrier\ \rcvp) \reduce{m}
  ((\thread_0\ \ldots\ ((\wait\ \aid_w)\ \cmd_1\ \ldots\ \bot)\ \thread_2\ \ldots)\ \epsnd\ \eprcv^\prime\ \epwait^\prime\ \epbarrier\ \rcvp)
}
\and
\inferrule[Barrier Command 1]
{
  \epbarrier(\aid_0) = v_c \\ v_c < N_{proc} \\ \epbarrier^\prime = \epbarrier[\aid_0 \mapsto  v_c + 1]
}{
 ((\thread_0\ \ldots\ ((\barrier\ \aid_0)\ \cmd_1\ \ldots\ \bot)\ \thread_2\ \ldots)\ \epsnd\ \eprcv\ \epwait\ \epbarrier\ \rcvp) \reduce{m}
 ((\thread_0\ \ldots\ ((\barrier\ \aid_0)\ \cmd_1\ \ldots\ \bot)\ \thread_2\ \ldots)\ \epsnd\ \eprcv\ \epwait\ \epbarrier^\prime\ \rcvp)
}
\and
\inferrule[Barrier Command 2]
{
  \epbarrier(\aid_0) = N_{proc}
}{
 ((\thread_0\ \ldots\ ((\barrier\ \aid_0)\ \cmd_1\ \ldots\ \bot)\ \thread_2\ \ldots)\ \epsnd\ \eprcv\ \epwait\ \epbarrier\ \rcvp) \reduce{m}
 ((\thread_0\ \ldots\ (\cmd_1\ \ldots\ \bot)\ \thread_2\ \ldots)\ \epsnd\ \eprcv\ \epwait\ \epbarrier\ \rcvp)
}

\end{mathpar}}
\caption{Machine Reductions ($\reduce{m}$). }
\label{fig:machine}
\end{figure*}

The \emph{Sndi Command} in \figref{fig:machine} consumes sends. 
$\epsnd^\prime$\ is a new set, just like the old set $\epsnd$, only the new set maps the destination
endpoint $v_{to}$\ and the source endpoint $v_{frm}$ to the number $v_c + 1$ where $v_c$ is the content in the old set.
The set is also updated such that it maps $v_{to}$\ and $\ast$ to the number $v_i + 1$ where $v_i$ is the content in the old set. Note that the notation $\ast$ is a special source endpoint indicating any source.

The \emph{Rcvi Command} in \figref{fig:machine} consumes receives by updating the set \epwait. 
Similar to the rule \emph{Sndi Command}, \epwait\ merely adds a new record for the receive $x_0$ that is indexed by the 
wait that witnesses the completion of $x_0$.

The \emph{Wait (Rcvi) Command} operates in three ways. 
If the wait $\aid_w$ maps to an empty set in \epwait, indicating that no receives need to be completed by $\aid_w$, then $\aid_w$ is simply consumed. 
If the first receive $\aid_0$ in $\epwait(\aid_w)$ is stored in \rcvp\ where the notation $\in$ is used to indicate this condition, then $\aid_0$ is removed from \rcvp\ meaning that $\aid_0$ in \rcvp\ is reached. 
The last rule for wait checks whether the first receive $\aid_0$\ in $\epwait(x_w)$ is able to be consumed. The rule is only active if $\aid_0$ is not in \rcvp. The rule requires that there are more counted sends than counted receives with common source and destination endpoints and that there are more counted sends than counted receives for any preceding wildcard receives. The condition indicate that at least one send can be matched with $\aid_0$. If this condition holds, then the set \eprcv\ is updated with the counted receive and $\aid_0$ is then removed from \epwait. 

The \emph{Barrier Command} moves the barrier forward by its synchronization rule. It is assumed that the group of any barrier consists of all the processes in $\mathit{ctp}$. 
If the count of the witnessed barriers $\epbarrier(\aid_0)$ for a specific communicator $\aid_0$ is less than the total number of the processes $N_{proc}$ indicating that the barriers for $\aid_0$ are not matched, then the barrier is not consumed and $\epbarrier(\aid_0)$ is incremented. The barrier can only be consumed when the count $\epbarrier(\aid_0)$ is equal to $N_{proc}$.

The machine rewrites the state until no more reduction rules can be applied indicating that there is no way to further execute the program. The first statement that cannot be consumed on any process is either the bottom of the process or a blocking command. A blocking command could be a wait or a barrier. If at the end, there are receive IDs in the pattern instance, then the pattern instance is provably non-feasible. In such a case, the machine \emph{accepts} the program as free of deadlock on the pattern instance; otherwise, the machine \emph{rejects} the program as having a deadlock on the pattern instance.

\begin{lemma}
  The machine implementing \textrm{FEASIBLECHECK} is sound in that it
  only accepts programs that do not deadlock on the associated pattern
  instance (but it may reject some programs as having a deadlock on
  that instance when in fact they do not).
\label{lemma:sound}
\end{lemma}
\begin{proof}
  Only \emph{Wait (RCVI) Command 3} is discussed since the
  other rules are simple (and trivial to prove). \emph{Wait (RCVI) Command 3} must never
  claim to not be able to consume a receive command incorrectly for
  the machine to be sound.
\end{proof}

\begin{cor}
  The machine implementing \textrm{FEASIBLECHECK} never miss counts the number of send and receives matched: \eprcv\ and \epsnd\ are correct.
\end{cor}
\begin{proof}
  \emph{Sndi Command} increments count every time the rule activates and updates both the counter for the specified endpoints and special counter that records sends that can match with wildcard receives. The \emph{Wait (Rcvi) Command 3} rule only increments the indicated receive counter when it is possible to consume the receive and by \lemmaref{lemma:sound}, the machine only matches in a way that is sound. The counters are not incremented in any other rule. Therefore, nothing is missed, and nothing is double counted.
\end{proof}
\begin{cor}
  \algoref{algo:main} is sound and complete if and only if \textrm{PATTERNMATCH} is complete (i.e., it gives all pattern instances and possibly more) and validate is sound and complete.
\end{cor}


\begin{comment}
%%Soundness Proof needs to be revised such that it shows the feasible check (operational semantics) is sound.
The completeness proof for the abstract machine in \figref{fig:machine} is given in \lemmaref{lemma:complete}.
This proof assumes that the function \textrm{PATTERNMATCH} detects all the instances for a particular pattern. This assumption is proved for the circular dependency pattern and the orphaned receive pattern latter sections. 

\begin{lemma}[Completeness for Feasible Check]
  The machine implementing \textrm{FEASIBLECHECK} is sound for a given pattern instance in that only a program with no deadlock on that instance is accepted; thus, it may reject some programs as having a deadlock on an instance when in fact they do not.
  
  The machine implementing \textrm{FEASIBLECHECK} is sound for a given pattern instance in that it may reject a program as having a deadlock on that instance when in fact it does not.
  
For any single-path MPI program, \textit{ctp}, any feasible schedule for a deadlock pattern instance is demonstrated by the function \textrm{FEASIBLECHECK} in \algoref{algo:main}. 
\label{lemma:complete}
\end{lemma}
\begin{proof}
Proof by showing that the abstract machine in \figref{fig:machine} simulates the message communication under infinite buffer semantics. For the \emph{Sndi Command} and \emph{Rcvi Command} rules, a send or receive is consumed immediately and two structures $\epsnd$ and $\epwait$, respectively, are updated. This is consistent with the issuing of send and receive under infinite buffer semantics. The three cases of \emph{Wait Command} witness the completion of the receives that are not in the pattern $\mathit{pt}$ and intend to get to the receives in $\mathit{pt}$. The two cases of \emph{Barrier Command} block the execution of a member process until all the barriers in the group are witnessed. Since the abstract machine in \figref{fig:machine} is able to simulate the behavior for infinite buffer semantics, any feasible schedule should be demonstrated by executing \textit{ctp} on the machine.
\end{proof}
\end{comment}

\section{Circular Dependency}

%%%% This section presents the pattern match algorithm and validation for circular dependency.


A deadlock may occur when there exists a circular dependency in messages. This circular dependency is defined in \defref{def:circular} that depends on the sequential relation defined in \defref{def:seqrelation}.

\begin{definition}
A sequential relation for a process, $p$, is a three-tuple $(p, \mathit{r_c}, \mathit{s_l})$, where the receive $\mathit{r_c}$ and the wait $\mathit{w_d}$ that witnesses the completion of $\mathit{r_c}$ are both followed by the send $\mathit{s_l}$ on $p$. 
\label{def:seqrelation}
\end{definition}

\begin{definition}
Given a set of sequential relations, $D$, a circular dependency $\tau$ $=$ $\langle(p_0, \mathit{r}_0, \mathit{s}_0),$ $\ldots,$ $(p_m, \mathit{r}_m, \mathit{s}_m)\rangle$ is a sequence in $D$, such that the following properties hold.
\begin{compactenum}
\item at least two sequential relations exist in $\tau$;
\item for all distinct $i,j \in [0,m]$, $p_i \neq p_j$;
\item for all $i \in [0,m], j = (i+1) \% m$, $\mathit{s}_i$ can potentially match $\mathit{r}_j$;
%\item for all $i \in [1,m]$, if $\mathtt{r}_i$ is a deterministic receive, then there are no wildcard receives preceding $\mathtt{r}_i$ on $p_i$.
\end{compactenum}
\label{def:circular}
\end{definition}

In \figref{fig:deadlock2}, the CTP has a circular dependency in messages $\langle(p_0, \mathit{r_{0}}, \mathit{s_{1}}),$ $(p_1, \mathit{r_{2}}, \mathit{s_{2}}),$ $(p_2, \mathit{r_{3}}, \mathit{s_{3}})\rangle$. Each receive in the circular dependency waits for the issuing of a send from one of the other processes but never gets a response. For instance, $\mathit{r_{0}}$ can never match $\mathit{s_{3}}$ because of the dependency. For simplicity, a instance of the circular dependency pattern only records the receives in a set. For example, $(\mathit{r_{0}}, \mathit{r_{2}}, \mathit{r_{3}})$ is a instance of the circular dependency pattern for the CTP in \figref{fig:deadlock2}. 

\begin{algorithm}
\caption{Finding Circular Dependency}\label{algo:circular}
\begin{algorithmic}[1]
%\Require $\mathit{ctp}$, a single-path MPI program
%\Require $\mathit{M}(\mathtt{r_c}) = \{\mathtt{s_l}\mid(\mathtt{s_l},\mathtt{r_c})\in\mathit{M}\}$, a set of potentially matched sends for $\mathtt{r_c}$
\State $\mathit{PT} \gets \emptyset$
\State $\mathit{E} \gets \emptyset$, $\mathit{V} \gets \emptyset$
%a relation from an operation $\mathit{e}$ to another operation $\mathit{e}$
%a set of operations
\State $\mathit{R} \gets \emptyset$, $\mathit{R_w} \gets \emptyset$
%\State $\mathit{PT}$, a set of circular dependency pattern instances
%\State $\mathit{R}(p) = \{\mathtt{r_c}\mid(p,\mathtt{r_c})\in\mathit{R}\}$, the witnessed receives on process $p$
%\State $\mathit{R_w}(p) = \{\mathtt{r_c}\mid(p,\mathtt{r_c})\in\mathit{R_w}\}$, the receives for the last witnessed wait on process $p$ 
\For{$(e_1, e_2, \dots, e_n) \in \mathit{ctp}$}
\For{$i \gets 1$ to $n$}
\If{$\mathit{e_i}$ is a receive}
%\If{$\mathit{isIssued}$ and $\mathtt{e}$ is a deterministic receive}
%\State \textbf{break}
%\EndIf
%\If{$\mathtt{e}$ is a wildcard receive}
%\State $\mathit{isIssued} \gets true$
%\EndIf
\State $\mathit{R} = \mathit{R} \cup \{\mathit{e_i}\}$
\State $\mathit{V} = \mathit{V} \cup \{\mathit{e_i}\}$
\For{$\mathit{s_l} \in \mathit{M}(\mathit{e_i})$} \Comment{add match relation to $\mathit{E}$}
\State $\mathit{E} = \mathit{E} \cup \{(\mathit{e_i},\mathit{s_l})\}$  
\EndFor
\EndIf
\If{$\mathit{e_i}$ is a wait}
\For{$\mathit{r_c} \in \mathit{R}$}
\State $\mathit{R_w} = \mathit{R_w} \cup \{\mathit{r_c}\}$
\EndFor
\State $\mathit{R} \gets \emptyset$ 
\EndIf
\If{$\mathit{e_i}$ is a send}
\State $\mathit{V} = \mathit{V} \cup \{\mathit{e_i}\}$
\For{$\mathit{r_c} \in \mathit{R_w}$}  \Comment{add HB relation from $\mathit{r_c}$ to $\mathit{e_i}$ to $\mathit{E}$}
\State $\mathit{E} = \mathit{E} \cup \{(\mathit{r_c},\mathit{e_i})\}$
\EndFor
\EndIf
\EndFor
\State $\mathit{R} \gets \emptyset$, $\mathit{R_w} \gets \emptyset$
\EndFor
\State $\mathit{PT} \gets$ \Call {Johnson}{$\mathit{V}$, $\mathit{E}$}
\end{algorithmic}
\end{algorithm}

\algoref{algo:circular} shows the steps for finding all the instances of the circular dependency pattern in $\mathit{ctp}$. In general, the algorithm is part of the function $\mathrm{PATTERNMATCH}$ in \algoref{algo:main}. It builds a graph by transforming the operations in $\mathit{ctp}$ to the vertices $\mathit{V}$ and transforming two types of relations to the edges $\mathit{E}$ and then finding all the cycles in the graph based on the Johnson's algorithm \cite{DBLP:journals/siamcomp/Johnson75}.
 
The variable $\mathit{V}$ is a set of operations. The variable $\mathit{E}$ is a set of edges each representing a relation from one operation to another. 
The set $\mathit{PT}$ stores the matched pattern instances.
The set $\mathit{R}$ stores the witnessed receives on the process $p$ that is a list of operations $(e_1, e_2, \dots, e_n)$. 
The set $\mathit{R_w}$ stores the receives that are recently witnessed on $p$.
%Each process $p$ in $\mathit{ctp}$ is a list of operations $(e_1, e_2, \dots, e_n)$. 

The algorithm checks each operation $e_i$ in $p$. 
If $\mathit{e_i}$ is a receive, then it is inserted into $\mathit{R}$ and $\mathit{V}$, respectively. Also, for each potentially matched send $\mathit{s_l}$ in $\mathit{M}(\mathit{e_i})$, the pair $(\mathit{e},\mathit{s_l})$ is inserted into $\mathit{E}$ indicating that a match relation from $\mathit{s_l}$ to $\mathit{e_i}$ is an edge in the graph. If $\mathit{e_i}$ is a wait, all the receives in $\mathit{R}$ are inserted to $\mathit{R_w}$ and $\mathit{R}$ is set back to an empty set, indicating that all the receives in $\mathit{R}$ are witnessed by $\mathit{e_i}$. \defref{def:seqrelation} requires for each sequential relation a receive and the wait that witnesses the completion of this receive both precede a send on an identical process. Therefore, %only the receives in the set $\mathit{Wset}$ are considered to build the happens-before relation. 
if $\mathit{e_i}$ is a send, the pair $(\mathit{r_c},\mathit{e_i})$ is inserted into $\mathit{E}$ at line 22 indicating that a happens-before relation from each witnessed receive $\mathit{r_c}\in\mathit{R_w}$ to $\mathit{e_i}$ is an edge in the graph. 

Finally, the function $\mathrm{JOHNSON}$ implements the Johnson's algorithm to compute all the cycles in the graph. Since an edge in any computed cycle is either a receive-send HB relation on an identical process, or a send-receive match relation across two processes, there exists exactly one sequential relation for any process in the cycle. Therefore, each cycle computed by the function $\mathrm{JOHNSON}$ is a instance of the circular dependency pattern. As discussed earlier, only receives in each cycle are stored to represent a instance in the set $\mathit{PT}$. The complexity of program traversal is $O(\mathrm{N}^2)$, where $\mathrm{N}$ is the total number of operations in the program. The complexity of Johnson's algorithm is $O((\mathrm{v}+\mathrm{e})(\mathrm{c}+1))$ $\approx$ $O((\mathrm{c}+1)\mathrm{N}^2)$, where $\mathrm{v}$ is the number of vertices, $\mathrm{e}$ is the number of edges, and $\mathrm{c}$ is the number of cycles. Therefore, the total complexity of the algorithm is $O((\mathrm{c}+1)\mathrm{N}^2)$.

%%revise the following algorithm of validating circular dependency pattern
\examplefigfour


If the function \textrm{FEASIBLECHECK} demonstrates that a potentially feasible schedule exists, \algoref{algo:vcircular} further validates whether or not a real deadlock exists for the pattern instance \textit{pt}. The condition at line 4 checks if there exists a send that may match any receive $\mathit{r_c}$ in \textit{pt} by comparing the count of issued sends and the count of issued receives.    
The function $\mathit{N_s}(\mathit{dest},\mathit{src})$ returns the count of issued sends with the destination $\mathit{dest}$ and the source $\mathit{src}$. The function $\mathit{N_r}(\mathit{dest},\mathit{src})$ returns the count of issued receives with the destination $\mathit{dest}$ and the source $\mathit{src}$.
If the condition is true, no real deadlock exists for \textit{pt} because $\mathit{r_c}$ can be matched and the cycle in \textit{pt} does not hold. If this condition is not satisfiable for any receive in \textit{pt}, the algorithm detects a real deadlock for \textit{pt}.
As an example, \algoref{algo:vcircular} reports that the CTP in \figref{fig:nodeadlock2} has no real deadlock for the instance of the circular dependency pattern $(r_0,r_1,r_2)$ because the send $s_0$ matches the receive $r_1$ and the cycle does not hold.

\begin{algorithm}
\caption{Validate Circular Dependency}\label{algo:vcircular}
\begin{algorithmic}[1]
%\Require $\mathit{ctp}$, a single-path MPI program
%\Require $\mathit{M}(\mathtt{r_c}) = \{\mathtt{s_l}\mid(\mathtt{s_l},\mathtt{r_c})\in\mathit{M}\}$, a set of potentially matched sends for $\mathtt{r_c}$
%\State  $\mathit{PT}$, a set of pattern instances
%\State  $\mathit{pt}$, a set of receives in the pattern instance $\mathit{pt}\in\mathit{PT}$
%\State $\mathit{N_s}$, a set of numbers each representing the the number of issued sends given a destination and a source
%\State $\mathit{N_r}$, a set of numbers each representing the number of issued receives given a destination and a source
%\State ($\mathit{ctp}_s, \mathit{N_s}, \mathit{N_r}, \mathit{empty}_{pt}) \gets$ \Call {ScheduleFinder}{$\mathit{pt}$}
%\If{$\mathit{empty}_{pt}$} 
%\If{\Call{isCircular}{$\mathit{pt}$}}
\For{$\mathit{r_c}\in\mathit{pt}$}
\State $\mathit{src} \gets$ source endpoint of $\mathit{r_c}$ 
\State $\mathit{dest} \gets$ destination endpoint of $\mathit{r_c}$
\If{$\mathit{N_s}(\mathit{dest},\mathit{src}) > \mathit{N_r}(\mathit{dest},\mathit{src})$}
\State report no deadlock and exit.
\EndIf
\EndFor
\State report deadlock.
%\State report deadlock and exit.
%\EndIf
%\If{\Call {isMismatch}{$\mathit{pt}} \in$}
%\If{\textproc{SAT}({\Call {Encode}{$\mathit{ctp}_s$}})}
%\State report deadlock and exit.
%\EndIf
%\EndIf
%\EndIf
%\EndProcedure
\end{algorithmic}
\end{algorithm}


%%Soundness proof for pattern match needs to be added. the following proof needs to be revised to show that the validation is sound and complete.

\begin{lemma}[Soundness for Circular Dependency Pattern]
For any message passing program, \textit{ctp}, the existence of a deadlock reported by \algoref{algo:main} for an instance of the circular dependency pattern, $\mathit{pt}$, indicates a real deadlock. 
\label{lemma:circular}
\end{lemma}
\begin{proof}
Proof by showing the existence of a deadlock. First, assume $(\mathit{ctp}^\prime,$ $\mathit{ctp}_s,$ $\mathit{N_s},$ $\mathit{N_r},$ $\mathit{P_r},$ $\mathit{P_b},$ $\mathit{R_{pt}}^\prime)$ is the machine state after applying the operational semantics in \figref{fig:machine} and no reduction commands can be further applied. Second, assume \algoref{algo:main} reports the existence of a deadlock at line $15$. Assume $\mathit{ctp}_s$ is a feasible schedule. Similar to the proof of \lemmaref{lemma:mismatch}, $\mathit{ctp}_s$ is executable in the runtime system. Also, any receive in $\mathit{pt}$ has no way to be matched if the condition at line $11$ in \algoref{algo:main} is not satisfied for any receive in $\mathit{pt}$. Further, for any process other than the processes in $\mathit{pt}$, the first operation is blocked in $\mathit{ctp}^\prime$ according to the operational semantics in \figref{fig:machine}. As such, $\mathit{ctp}^\prime$ deadlocks. As such, the program \textit{ctp} may deadlock for the pattern instance $\mathit{pt}$ in program execution. 
\end{proof} 

%%zero buffer for circular dependency

\subsection{Deadlock for Zero Buffer Semantics}
The deadlock patterns discussed above may also cause a program to deadlock under zero buffer semantics. Notice that the commands in \figref{fig:machine} are consistent with how messages communicate under infinite buffer semantics (e.g., a send is consumed immediately). The zero buffer semantics, however, enforce a different way of message communication. As such, a feasible infinite buffer schedule generated by \algoref{algo:main} is not able to witness a deadlock under zero buffer semantics. Therefore, the zero buffer compatibility should be checked for each generated schedule. \algoref{algo:main} is extended with two changes: 1) the zero buffer encoding rules (refer to \cite{HuangNFM15}) are added to the function $\mathrm{ENCODE}$; 2) The circular dependency pattern also needs an SMT encoding to check the feasibility of a schedule. As a note, to encode the match pairs for the circular dependency pattern only needs to ensure that each receive in the schedule is matched with some send. Once the schedule is proved to be zero buffer compatible, the program may deadlock under zero buffer semantics; otherwise, the pattern instance does not imply a real deadlock. 



\section{Orphaned Receive}

%% rename the pattern as orphaned receive. This section presents the pattern match algorithm and validation for orphaned receive.

\examplefigone

A deadlock may occur when the runtime matches a wildcard receive with the send needed by a deterministic receive. 

\figref{fig:deadlock1} is an example CTP that deadlocks for an orphaned receive pattern instance. The underlined command indicates a pattern instance. Process $p_0$ sends a message to $p_1$, and then receives two messages from any source and $p_1$ in the receive $r_0$ and $r_1$. Process $p_1$ receives a message from any source, and then sends a message to $p_0$. Process $p_2$ sends a message to $p_0$.  
A deadlock may occur if the receive $\mathit{r_{0}}$ is matched with the send $\mathit{s_{1}}$ making it unable to match the receive $\mathit{r_{1}}$ with the send $\mathit{s_{2}}$. 
This CTP shows an instance of the orphaned receive pattern in \defref{def:mismatch}. 

\begin{definition}
An orphaned receive is a pair $(\mathit{r_w}, \mathit{r_c})$ where, 
\begin{compactenum}
\item $\mathit{r_w}$ is a wildcard receive and $\mathit{r_c}$ is a deterministic receive that follows $\mathit{r_w}$ on an identical process $\mathit{p}$;
\item at least two sends from different source endpoints other than $\mathit{p}$ may potentially match $\mathit{r_w}$; and
\item among these matched sends, at least one send matches $\mathit{r_c}$. 
\end{compactenum}
\label{def:mismatch}
\end{definition}

For simplicity, only the deterministic receive is stored to represent an instance of the orphaned receive pattern. For example, the CTP in \figref{fig:deadlock1} has an orphaned receive pattern instance $(r_1)$.
As a note, the presentation in this section is only for infinite buffer semantics in the discussion of algorithms and examples. The zero buffer semantics are discussed in section 6.

\subsection{Pattern Match for Orphaned Receive}

\begin{algorithm}
\caption{Finding Orphaned Receive}\label{algo:mismatch}
\begin{algorithmic}[1]
%\Require $\mathit{ctp}$, a single-path MPI program
%\Require $\mathit{M}(\mathtt{r_c}) = \{\mathtt{s_l}\mid(\mathtt{s_l},\mathtt{r_c})\in\mathit{M}\}$, a set of potentially matched sends for $\mathtt{r_c}$
%\State $\mathit{PT}$, a set of mismatched send-receive pattern instances
%\State $\mathit{R}(p) = \{\mathtt{r_w}\mid(p,\mathtt{r_w})\in\mathit{R}\}$, a set of wildcard receives on process $p$
\State $\mathit{PT} \gets \emptyset$
\State $\mathit{R} \gets \emptyset$
\For{$(e_1, e_2, \dots, e_n) \in \mathit{ctp}$}
\For{$i \gets 1$ to $n$}
\If{$e_i$ is a wildcard receive}
\State $\mathit{R} = \mathit{R} \cup \{e_i\}$
\EndIf
\If{$e_i$ is a deterministic receive}
\For{$\mathit{r_w} \in \mathit{R}$}
\If{$\mathit{M}(e_i) \cap \mathit{M}(\mathit{r_w}) \neq \emptyset \wedge |\mathit{M}(\mathit{r_w})| > 1$}
\State $\mathit{PT} = \mathit{PT} \cup \{\{e_i\}\}$
%\State \textbf{break}
\EndIf
\EndFor
\EndIf
\EndFor
\State $\mathit{R} \gets \emptyset$
\EndFor
%\EndProcedure
\end{algorithmic}
\end{algorithm}

\algoref{algo:mismatch} shows the steps of finding the orphaned receive pattern instances for an input \textit{ctp}. The algorithm is part of the function $\mathrm{PATTERNMATCH}$ in \algoref{algo:main}.  
It checks all the pairs of wildcard/deterministic receives on each process, and then finds all the instances that satisfy the criteria in \defref{def:mismatch}. 

The set $\mathit{PT}$ stores the matched pattern instances.
The set $\mathit{R}$ stores the witnessed wildcard receives on the process \textit{p} that is a list of operations $(e_1, e_2, \dots, e_n)$. 

The algorithm checks each operation $e_i$ in $p$. If $e_i$ is a wildcard receive, then it is inserted into $\mathit{R}$ indicating that it is witnessed. \defref{def:mismatch} requires that the wildcard receive in an orphaned receive pair happens before the deterministic receive in the same pair on an identical process. Therefore, only a wildcard receive with a following deterministic receive is able to build an orphaned receive pattern instance. If $e_i$ is a deterministic receive, it checks each wildcard receive $r_w$ in $\mathit{R}$. If the condition at line 10 is also satisfied such that $r_w$ and $e_i$ have a common set of potential sends and more than one send can be matched with $r_w$, a new instance $\{e_i\}$ is added to $\mathit{PT}$ at line 11. $R$ is set back to empty at line 17, indicating that the algorithm starts at a new process for the pairs of wildcard/deterministic receives.
The complexity of the algorithm is $O(\mathrm{N}^2)$, where $\mathrm{N}$ is the total number of operations in the \textit{ctp}. 

The completeness for \algoref{algo:mismatch} is given in \lemmaref{lemma:pmorphaned}. 

\begin{lemma}
\label{lemma:pmorphaned}
The pattern match in \algoref{algo:mismatch} is complete indicating that all possible pattern instances are detected (it may include some instances that do not have deadlocks).
\end{lemma}
\begin{proof}
\algoref{algo:mismatch} considers all possible pairs of wildcard/deterministic receives on each process. Also, the pairs that do not satisfy the condition at line 10 only contains those never cause the deterministic receives orphaned, because 1) there does not exist any common send that can potentially match the wildcard receive and the deterministic receive, or 2) there is no non-deterministic choice for the wildcard receive to be matched with. 
Therefore, all possible instances of the orphaned receive pattern are detected. 
\end{proof}

%%add text for the following algorithm of validating orphaned receive pattern
\subsection{Validation for Orphaned Receive}

\begin{algorithm}
\caption{Validate Orphaned Receive}\label{algo:vorphaned}
\begin{algorithmic}[1]
%\Require $\mathit{M}(\mathtt{r_c}) = \{\mathtt{s_l}\mid(\mathtt{s_l},\mathtt{r_c})\in\mathit{M}\}$, a set of potentially matched sends for $\mathtt{r_c}$
%\State  $\mathit{ctp_s}$, a potential schedule detected by the Call {FeasibleCheck}{}{}
\If{\textproc{SAT}({\Call {Encode}{$\mathit{ctp}_s$, $\mathit{pt}$, $\mathit{M}$}})}
\State report deadlock and exit.
\EndIf
\end{algorithmic}
\end{algorithm}

The validation in \algoref{algo:vorphaned} is more complicated than the validation for the circular dependency because it requires a higher cost SMT encoding for the $\mathit{ctp_s}$ that is generated by a modification of the abstract machine in \figref{fig:machine}.
This modification executes a CTP just like the old machine, only it updates a new $\mathit{ctp_s}$ by adding the consumed operations when executing the machine.
The function $\mathrm{ENCODE}$ encodes the $\mathit{ctp_s}$ into an SMT problem based on the rules in \cite{DBLP:conf/kbse/HuangMM13}. Further, a new rule is added to the encoding: for the receive $\mathit{r_c}$ in the pattern instance \textit{pt}, $\bigvee_{\mathit{r_i}\in\mathit{M}(\mathit{s_l})}\langle\mathit{r_i},\mathit{s_l}\rangle$ is encoded for any send $\mathit{s_l}\in\mathit{M}(\mathit{r_c})$ in $\mathit{ctp_s}$. This rule ensures that no send in the $\mathit{ctp_s}$ can match $\mathit{r_c}$. The function $\mathrm{SAT}$ is true if the encoding is satisfiable. The existence of a satisfying assignment of the encoding implies a real deadlock for \textit{pt}. If the encoding is unsatisfiable, no deadlock exists for \textit{pt}.

\examplefigtwo

The CTP in \figref{fig:nodeadlock1} is an example that has no deadlock for the orphaned receive pattern instance. In \figref{fig:nodeadlock1}, process $p_0$ receives a message from any source, sends a message to $p_1$, and then receives a message from $p_1$. Process $p_1$ receives a message from any source, and then sends a message to $p_0$. Process $p_2$ sends a message to $p_0$. As shown, the pair $(r_0,r_1)$ satisfies the criteria in \defref{def:mismatch}, therefore, the instance $\{r_1\}$ can be detected by \algoref{algo:mismatch}. However, the encoding generated by the function $\mathrm{ENCODE}$ in \algoref{algo:vorphaned} is unsatisfiable for the instance $\{r_1\}$ because the receive $r_0$ can not match the send $s_1$ so the receive $r_1$ is matched eventually.

%%Soundness proof for pattern match needs to be added. the following proof needs to be revised to show that the validation is sound and complete.
The soundness proof of the validation in \algoref{algo:vorphaned} is given in \lemmaref{lemma:mismatch}.

\begin{lemma}
The validation method implementing \algoref{algo:vorphaned} for an instance of the orphaned receive pattern, $\mathit{pt}$, is sound and complete indicating that any detected deadlock is a real deadlock and any instance rejected by \algoref{algo:vorphaned} is not a real deadlock. 
\label{lemma:mismatch}
\end{lemma}
\begin{proof}
The soundness and completeness relies on the correctness proof of the SMT encoding \cite{DBLP:conf/kbse/HuangMM13}. In general, a satisfying assignment returned by \algoref{algo:vorphaned} represents a feasible schedule leading to an orphaned receive. Therefore, a real deadlock is detected. If the SMT encoding is unsatisfiable indicating that no schedules can be resolved for $pt$, then no real deadlock may occur.
\end{proof}

%The SMT encoding used in the validation step for the orphaned receive pattern can be solved by a state-of-art SMT solver Z3 with a higher cost \cite{demoura:tacas08}. The general algorithm, however, can be modified to remove the validation for the orphaned receive pattern, which leads to a much faster deadlock detection. If so, the algorithm is not sound. The completeness, however, is maintained.

%%zero buffer for orphaned receive



%\section{Algorithm}

\examplefigone

The algorithm in this paper consists of several steps shown in \algoref{algo:main}. The function $\mathrm{PATTERNFINDER}$ at line 3 computes two types of deadlock patterns (defined later) by running \algoref{algo:mismatch} and \algoref{algo:circular} in sequence, where each statically traverses the program. The function $\mathrm{SCHEDULEFINDER}$ at line 5, again, statically traverses the program and finds a potential schedule $\mathit{ctp}_s$. The variable $\mathit{empty}_{pt}$ returned by $\mathrm{SCHEDULEFINDER}$ is true if all the receives in the pattern instance \textit{pt} are witnessed after the schedule is detected. \algoref{algo:main} further checks the feasibility of this schedule with different strategies for the two types of deadlocks. The functions $\mathrm{isCircular}$ and $\mathrm{isMismatch}$ check the types of deadlock patterns. If \textit{pt} is a circular dependency instance, the condition at line 11 in \algoref{algo:main} checks whether there exists a send in the schedule that may match any receive $\mathit{r}$ in \textit{pt}. If the condition is true for some $\mathit{r}$ in \textit{pt}, then the instance is not a real deadlock because $\mathit{r}$ is matched and the circle in \textit{pt} does not hold. Once all the receives in \textit{pt} are checked such that they can not be matched to any sends in the schedule, then the feasibility of the schedule is witnessed. If \textit{pt} is a mismatched send and receive instance, the function $\mathrm{ENCODE}$ in \algoref{algo:main} encodes the schedule into an SMT problem based on the rules in \cite{DBLP:conf/kbse/HuangMM13}. Further, a new rule is extended for the match pair encoding: for any deterministic receive $\mathtt{r_c}$ in\textit{pt}, $\bigvee_{\mathtt{r_i}\in\mathit{M}(\mathtt{s_l})}(\mathtt{s_l},\mathtt{r_i})$ is encoded for any $\mathtt{s_l}\in\mathit{M}(\mathtt{r_c})$ in the schedule. This rule ensures that no send in the schedule can match $\mathtt{r_c}$. The function $\mathrm{SAT}$ is true if the encoding is satisfiable. The existence of a satisfying assignment of the encoding implies that the schedule is feasible. The algorithm aborts the verification process once a deadlock is found at line 15 or line 19. As a note, the algorithm is only applied for verifying programs under infinite buffer semantics. If the zero buffer semantics are enforced, the algorithm needs to be modified as discussed later.

%The algorithm requires a single-path MPI program, $\mathit{ctp}$, as input. In the first step, the function $\mathrm{PATTERNFINDER}$ at line 4 traverses the program linearly and finds the instances of two types of deadlock patterns (defined below). In the second step, the function $\mathrm{SCHEDULEFINDER}$ at line 7, again, statically traverses the program and finds a potential schedule $\mathit{ctp}_s$ and updates the set $\rcvp^\prime$ for each pattern instance $\mathit{pt}$. If the new set $\rcvp^\prime$ is not an empty set (at line 8), or there are more sends than receives with identical $src$ and $dest$ that are the source and destination endpoints respectively for any receive in \rcvp\ (at line 12), then the algorithm continues to check the next pattern instance. It repeats the steps above until a deadlock is found or no deadlock exists for all the pattern instances. If a schedule exists, the algorithm reports a deadlock at line 17 if the pattern instance is a circular dependency, or further checks the feasibility of the schedule by encoding it into an SMT problem at line 20 if the pattern instance is a mismatched send-receive. The algorithm reports a deadlock at line 22 if the encoding is satisfiable, or continues to check the next pattern instance at line 25 if it is unsatisfiable. The algorithm aborts the verification process once a deadlock is found. As a note, the algorithm is only applied for verifying programs under infinite buffer semantics. To adjust zero buffer semantics, this paper presents how to modify the algorithm later.

\begin{algorithm}
\caption{Main Framework}\label{algo:main}
\begin{algorithmic}[1]
%\Procedure{Main Entrance}{}
\Require $\mathit{ctp}$, a single-path MPI program
\Require $\mathit{M}(\mathtt{r_c}) = \{\mathtt{s_l}\mid(\mathtt{s_l},\mathtt{r_c})\in\mathit{M}\}$, a set of potentially matched sends for $\mathtt{r_c}$
\State  $\mathit{PT}$, a set of pattern instances
\State  $\mathit{pt}$, a set of receives in the pattern instance $\mathit{pt}\in\mathit{PT}$
\State  $\mathit{PT} \gets$ \Call {PatternFinder}{$\mathit{ctp}$, $\mathit{M}$}
\State \emph{point:}
\For{$\mathit{pt} \in \mathit{PT}$}
\State ($\mathit{ctp}_s, \mathit{N_s}, \mathit{N_r}, \mathit{empty}_{pt}) \gets$ \Call {ScheduleFinder}{$\mathit{pt}$}
\If{$\mathit{empty}_{pt}$} 
\If{\Call{isCircular}{$\mathit{pt}$}}
\For{$\mathtt{r_c}\in\mathit{pt}$}
\State $\mathit{src} \gets$ source endpoint of $\mathtt{r_c}$, $\mathit{dest} \gets$ destination endpoint of $\mathtt{r_c}$
\If{$\mathit{N_s}(\mathit{dest},\mathit{src}) > \mathit{N_r}(\mathit{dest},\mathit{src})$}
\State \textbf{continue} \textit{point}
\EndIf
\EndFor
\State report deadlock and exit.
\EndIf
\If{\Call {isMismatch}{$\mathit{pt}} \in$}
\If{\textproc{SAT}({\Call {Encode}{$\mathit{ctp}_s$}})}
\State report deadlock and exit.
\EndIf
\EndIf
\EndIf
\EndFor
%\EndProcedure
\end{algorithmic}
\end{algorithm}

\subsection{Finding Pattern Instances}

A deadlock may occur if sends and receives are mismatched because of the existence of deterministic receives. For instance, in \figref{fig:deadlock1}, a deadlock may occur if $\mathtt{s_{12}}$ matches $\mathtt{r_{01}}$ and there is no way to match $\mathtt{s_{20}}$ with $\mathtt{r_{03}}$. This program shows an instance of the mismatched send-receive pattern that is a set of the underlined operations. Only the deterministic receive $\mathtt{r_{03}}$ is stored to represent the pattern instance defined in \defref{def:mismatch}.


\begin{definition}
A mismatched send and receive pattern is matched when there exists a deterministic receive $\mathtt{r_c}$ where, 
\begin{compactenum}
\item there exists a wildcard receive $\mathtt{r_w}$ followed by $\mathtt{r_c}$ on a single process $\mathit{p}$; 
\item at least two sends from different source endpoints other than $\mathit{p}$ may potentially match $\mathtt{r_w}$; and
\item among these matched sends, at least one send matches $\mathtt{r_c}$. 
\end{compactenum}
\label{def:mismatch}
\end{definition}

\examplefigtwo

The existence of a pattern instance does not mean a deadlock occurs. The program in \figref{fig:nodeadlock1} is such an instance. This pattern instance is detected, but is pruned at line $17$ in \algoref{algo:main} because there does not exist a feasible schedule (a schedule that is allowed by MPI semantics).



\begin{algorithm}
\caption{Finding Mismatched Send-Receive}\label{algo:mismatch}
\begin{algorithmic}[1]
\Require $\mathit{ctp}$, a single-path MPI program
\Require $\mathit{M}(\mathtt{r_c}) = \{\mathtt{s_l}\mid(\mathtt{s_l},\mathtt{r_c})\in\mathit{M}\}$, a set of potentially matched sends for $\mathtt{r_c}$
\State $\mathit{PT}$, a set of mismatched send-receive pattern instances
\State $\mathit{R}(p) = \{\mathtt{r_w}\mid(p,\mathtt{r_w})\in\mathit{R}\}$, a set of wildcard receives on process $p$
\For{$p \in \mathit{ctp}$}
\For{$\mathtt{r_c} \in p$}
\If{$\mathtt{r_c}$ is a wildcard receive}
\State $\mathit{R} = \mathit{R} \cup \{(p,\mathtt{r_c})\}$
\EndIf
\If{$\mathtt{r_c}$ is a deterministic receive}
\For{$\mathtt{r_w} \in \mathit{R}(p)$}
\If{$\mathit{M}(\mathtt{r_c}) \cap \mathit{M}(\mathtt{r_w}) \neq \emptyset \wedge |\mathit{M}(\mathtt{r_w})| > 1$}
\State $\mathit{PT} = \mathit{PT} \cup \{\{\mathtt{r_c}\}\}$
\State \textbf{break}
\EndIf
\EndFor
\EndIf
\EndFor
\EndFor
%\EndProcedure
\end{algorithmic}
\end{algorithm}

\algoref{algo:mismatch} shows the steps for finding the mismatched send and receive pattern instances for an MPI program. In general, the algorithm is part of the function $\mathrm{PATTERNFINDER}$ in \algoref{algo:main}. It requires a single-path MPI program $\mathit{ctp}$ and the preprocessed match pair set $\mathit{M}$ as input. A match pair is a coupling of a receive and a send that may match in the runtime system.
%For each process $p$ in the program, the algorithm checks each receive $\mathtt{R}$. If $\mathtt{R}$ is a wildcard receive, then it is inserted into the set $\mathit{Rset}[p]$. if it is a deterministic receive, then the algorithm checks all the stored wildcard receives in $\mathit{Rset}[p]$. If there exists a wildcard receive $\mathtt{R_w} \in \mathit{Rset}[p]$ such that $\mathtt{R}$ and $\mathtt{R_w}$ have at least two common sends, then $\mathtt{R}$ is inserted into the set of mismatched send-receive pattern instances $\mathit{Pset}$. 
The complexity of the algorithm is $O(\mathrm{N}^2)$, where $\mathrm{N}$ is the total number of operations in the program. 
%Take the program in \figref{fig:deadlock1} as an example. \algoref{algo:mismatch} traverses the process $\mathtt{P0}$ and detects that the wildcard receive $\mathtt{R_{0,1}}$ is followed by the deterministic receive $\mathtt{R_{0,3}}$ and they have two common potential matched sends $\mathtt{S_{1,2}}$ and $\mathtt{S_{2,0}}$. Therefore, $\mathtt{R_{0,3}}$ satisfies the requirement of a mismatched send-receive pattern instance. 

A deadlock may also occur when there exists a circular dependency in messages. This circular dependency is defined in \defref{def:circular} that depends on the sequential relation defined in \defref{def:seqrelation}.

\begin{definition}
A sequential relation for a process, $p$, is a three-tuple $(p, \mathtt{r_c}, \mathtt{s_l})$, where the receive $\mathtt{r_c}$ and its nearest-enclosing wait $\mathtt{w_d}$ are both followed by a send $\mathtt{s_l}$ on $p$. 
\label{def:seqrelation}
\end{definition}

\begin{definition}
Given a set of sequential relations, $D$, a circular dependency pattern $\tau$ $=$ $\langle(p_0, \mathtt{r}_0, \mathtt{s}_0),$ $\ldots,$ $(p_m, \mathtt{r}_m, \mathtt{s}_m)\rangle$ is a sequence in $D$, such that the following properties hold.
\begin{compactenum}
\item at least two sequential relations exist in $\tau$;
\item for all distinct $i,j \in [0,m]$, $p_i \neq p_j$;
\item for all $i \in [0,m], j = (i+1) \% m$, $\mathtt{s}_i$ can potentially match $\mathtt{r}_j$;
%\item for all $i \in [1,m]$, if $\mathtt{r}_i$ is a deterministic receive, then there are no wildcard receives preceding $\mathtt{r}_i$ on $p_i$.
\end{compactenum}
\label{def:circular}
\end{definition}

\examplefigthree

\figref{fig:deadlock2} shows an MPI program with a circular dependency in messages $\langle(p_0, \mathtt{r_{01}}, \mathtt{s_{03}}),$ $(p_1, \mathtt{r_{12}}, \mathtt{s_{14}}),$ $(p_1, \mathtt{r_{20}}, \mathtt{s_{22}})\rangle$. Each receive in the pattern instance waits for the issuing of a send from one of the other processes but never gets a response. For instance, $\mathtt{r_{01}}$ can never match $\mathtt{s_{22}}$ because of the circular dependency. For simplicity, only receives are stored to represent a circular dependency pattern instance (e.g., $(\mathtt{r_{01}}, \mathtt{r_{12}}, \mathtt{r_{20}})$ is a pattern instance for the program in \figref{fig:deadlock2}). 

\begin{algorithm}
\caption{Finding Circular Dependency}\label{algo:circular}
\begin{algorithmic}[1]
\Require $\mathit{ctp}$, a single-path MPI program
\Require $\mathit{M}(\mathtt{r_c}) = \{\mathtt{s_l}\mid(\mathtt{s_l},\mathtt{r_c})\in\mathit{M}\}$, a set of potentially matched sends for $\mathtt{r_c}$
\State $\mathit{E} \subseteq \mathit{e}\times\mathit{e}$, a relation from an operation $\mathit{e}$ to another operation $\mathit{e}$
\State $\mathit{V}$, a set of operations
%\State $\mathit{isIssued}$, a bool variable
\State $\mathit{PT}$, a set of circular dependency pattern instances
\State $\mathit{R}(p) = \{\mathtt{r_c}\mid(p,\mathtt{r_c})\in\mathit{R}\}$, the witnessed receives on process $p$
\State $\mathit{R_w}(p) = \{\mathtt{r_c}\mid(p,\mathtt{r_c})\in\mathit{R_w}\}$, the receives for the last witnessed wait on process $p$ 
\For{$p \in \mathit{ctp}$}
\For{$\mathit{e} \in p$}
\If{$\mathit{e}$ is a receive}
%\If{$\mathit{isIssued}$ and $\mathtt{e}$ is a deterministic receive}
%\State \textbf{break}
%\EndIf
%\If{$\mathtt{e}$ is a wildcard receive}
%\State $\mathit{isIssued} \gets true$
%\EndIf
\State $\mathit{R} = \mathit{R} \cup \{(p,\mathit{e})\}$
\State $\mathit{V} = \mathit{V} \cup \{\mathit{e}\}$
\For{$\mathtt{s_l} \in \mathit{M}(\mathit{e})$} \Comment{add match relations to $\mathit{E}$}
\State $\mathit{E} = \mathit{E} \cup \{(\mathit{e},\mathtt{s_l})\}$  
\EndFor
\EndIf
\If{$\mathit{e}$ is a wait}
\For{$\mathtt{r_c} \in \mathit{R}(p)$}
\State $\mathit{R_w} = \mathit{R_w} \cup \{(p,\mathtt{r_c})\}$
\EndFor
\State $\mathit{R}(p) \gets \emptyset$ 
\EndIf
\If{$\mathit{e}$ is a send}
\State $\mathit{V} = \mathit{V} \cup \{\mathit{e}\}$
\For{$\mathtt{r_c} \in \mathit{R_w}(p)$}  \Comment{add HB relations from $\mathtt{r_c}$ to $\mathit{e}$ to $\mathit{E}$}
\State $\mathit{E} = \mathit{E} \cup \{(\mathtt{r_c},\mathit{e})\}$
\EndFor
\EndIf
\EndFor
\EndFor
\State $\mathit{PT} \gets$ \Call {Johnson}{$\mathit{V}$, $\mathit{E}$}
\end{algorithmic}
\end{algorithm}

\algoref{algo:circular} shows the steps for finding all the instances of the circular dependency pattern in a single-path MPI program $\mathit{ctp}$. In general, the algorithm is part of the function $\mathrm{PATTERNFINDER}$ in \algoref{algo:main}. It builds a graph by transforming the operations in $\mathit{ctp}$ to the vertices $\mathit{V}$ and transforming two types of relations to the edges $\mathit{E}$. To be precise, for each process $p$ in $\mathit{ctp}$, the algorithm checks each operation $\mathit{e}$. 
%If $\mathtt{e}$ is a deterministic receive and there are wildcard receives preceding $\mathtt{e}$ on a single process (property $4$ in \defref{def:circular} does not hold), then $\mathtt{e}$ is not inserted into $\mathit{V}$ at line $11$. If $\mathtt{e}$ is a wildcard receive, then the variable $\mathit{isIssued}$ is set to true. 
If $\mathit{e}$ is a receive, then the pair $(p,\mathit{e})$ is inserted into $\mathit{R}$ and $\mathit{V}$, respectively. Also, for each potentially matched send $\mathtt{s_l}$ in $\mathit{M}(\mathit{e})$, the pair $(\mathit{e},\mathtt{s_l})$ is inserted into $\mathit{E}$ indicating that a match relation from $\mathtt{s_l}$ to $\mathit{e}$ is an edge in the graph. If $\mathit{e}$ is a wait, all the receives in $\mathit{R}(p)$ are inserted to $\mathit{R_w}(p)$ and $\mathit{R}(p)$ is set to an empty set, indicating that $\mathit{e}$ is the nearest-enclosing wait for the receives in $\mathit{R}(p)$. \defref{def:seqrelation} requires for each sequential relation a receive and its nearest-enclosing wait both precede a send on a single process. Therefore, %only the receives in the set $\mathit{Wset}$ are considered to build the happens-before relation. 
if $\mathit{e}$ is a send, then the pair $(\mathtt{r_c},\mathit{e})$ is inserted into $\mathit{E}$ at line $24$ indicating that a happens-before relation from each receive $\mathtt{r_c}\in\mathit{R_w}$ to $\mathit{e}$ is an edge in the graph. Finally, the function $\mathrm{JOHNSON}$ implements the Johnson's algorithm to compute all the cycles \cite{DBLP:journals/siamcomp/Johnson75}. Since an edge in any computed cycle is either a receive-send HB relation on an identical process, or is a send-receive match relation across two processes, there exists exactly one sequential relation for any process of the cycle. Therefore, each cycle computed by the function $\mathrm{JOHNSON}$ has potential to cause deadlock. Only receives in each cycle are stored as an instance of the circular dependency pattern in the set $\mathit{PT}$. The complexity of program traversal is $O(\mathrm{N}^2)$, where $\mathrm{N}$ is the total number of operations in the program. The complexity of Johnson's algorithm is $O((\mathrm{v}+\mathrm{e})(\mathrm{c}+1))$ $\approx$ $O((\mathrm{c}+1)\mathrm{N}^2)$, where $\mathrm{v}$ is the number of vertices, $\mathrm{e}$ is the number of edges, and $\mathrm{c}$ is the number of cycles. Therefore, the total complexity of the algorithm is $O((\mathrm{c}+1)\mathrm{N}^2)$.

\examplefigfour

An instance of the circular dependency pattern does not mean a deadlock occurs. For example, the program in \figref{fig:nodeadlock2} has an instance of the circular dependency that is detected by \algoref{algo:circular}. However, it is pruned by \algoref{algo:main} because there does not exist a feasible schedule.

\subsection{Static analysis}


\begin{figure*}[tb]
\centering
\scalebox{0.9}{
\mprset{flushleft}
\begin{mathpar}

\inferrule[Sndi Command]{
  \epsnd(v_{to},v_{frm}) = v_c \\ \epsnd^\prime = \epsnd[(v_{to},v_{frm}) \mapsto v_c +1] \\ \epsnd(v_{to},\ast) = v_i \\ \epsnd^{\prime\prime} = \epsnd^\prime[(v_{to},\ast) \mapsto v_i +1] 
}{
  ((\thread_0\ \ldots\ ((\sendi\ v_{frm}\ v_{to})\ \cmd_1\ \ldots\ \bot)\ \thread_2\ \ldots)\ (\thread_0^\prime\ \ldots\ (\cmd_2\ \ldots\ \bot)\ \thread_2^\prime\ \ldots)\ \epsnd\ \eprcv\ \epwait\ \epbarrier\ \rcvp)\\ \reduce{m}
  ((\thread_0\ \ldots\ (\cmd_1\ \ldots\ \bot)\ \thread_2\ \ldots)\ (\thread_0^\prime\ \ldots\ (\cmd_2\ \ldots\ (\sendi\ v_{frm}\ v_{to})\ \bot)\ \thread_2^\prime\ \ldots)\ \epsnd^{\prime\prime}\ \eprcv\ \epwait\ \epbarrier\ \rcvp)
}

\and

\inferrule[Rcvi Command]{
 \epwait(\aid_w) =  ([\aid_1\ v_{frm1}\ v_{to1}]\ \ldots)
 \\ \epwait^\prime = \epwait [ \aid_w \mapsto ([\aid_0\ v_{frm0}\ v_{to0}]\ [\aid_1\ v_{frm1}\ v_{to1}]\ \ldots])] 
}{
  ((\thread_0\ \ldots\ ((\recvi\ \aid_0\ v_{frm0}\ v_{to0}\ \aid_w)\ \cmd_1\ \ldots\ \bot)\ \thread_2\ \ldots)\ (\thread_0^\prime\ \ldots\ (\cmd_2\ \ldots\ \bot)\ \thread_2^\prime\ \ldots)\ \epsnd\ \eprcv\ \epwait\ \epbarrier\ \rcvp) \\ \reduce{m}
  ((\thread_0\ \ldots\ (\cmd_1\ \ldots\ \bot)\ \thread_2\ \ldots)\ (\thread_0^\prime\ \ldots\ (\cmd_2\ \ldots\ (\recvi\ \aid_0\ v_{frm0}\ v_{to0}\ \aid_w)\ \bot)\ \thread_2^\prime\ \ldots)\ \epsnd\ \eprcv\ \epwait^\prime\ \epbarrier\ \rcvp)
}
\and
\inferrule[Wait (rcvi) Command 1]
{
  \epwait(\aid_w) = ()
}{
  ((\thread_0\ \ldots\ ((\wait\ \aid_w\ \aid_0)\ \cmd_1\ \ldots\ \bot)\ \thread_2\ \ldots)\ (\thread_0^\prime\ \ldots\ (\cmd_2\ \ldots\ \bot)\ \thread_2^\prime\ \ldots)\ \epsnd\ \eprcv\ \epwait\ \epbarrier\ \rcvp)\\ \reduce{m}
  ((\thread_0\ \ldots\ (\cmd_1\ \ldots\ \bot)\ \thread_2\ \ldots)\ (\thread_0^\prime\ \ldots\ (\cmd_2\ \ldots\ (\wait\ \aid_w\ \aid_0)\ \bot)\ \thread_2^\prime\ \ldots)\ \epsnd\ \eprcv\ \epwait\ \epbarrier\ \rcvp)
}
\and
\inferrule[Wait (rcvi) Command 2]
{
   \epwait(\aid_w) = ([\aid_2\ v_{frm2}\ v_{to2}]\ [\aid_3\ v_{frm3}\ v_{to3}]\ \ldots\ [\aid_1\ v_{frm1}\ v_{to1}])\ \rcvp = (\aid_a\ \ldots\ \aid_b\ \aid_2\ \aid_c\ \ldots) \\ \aid_2 \in \rcvp \\ \rcvp^\prime = (\aid_a\ \ldots\ \aid_b\ \aid_c\ \ldots)
}{
  ((\thread_0\ \ldots\ ((\wait\ \aid_w\ \aid_0)\ \cmd_1\ \ldots\ \bot)\ \thread_2\ \ldots)\ ctp_s\ \epsnd\ \eprcv\ \epwait\ \epbarrier\ \rcvp)\\ \reduce{m}
  ((\thread_0\ \ldots\ ((\wait\ \aid_w\ \aid_0)\ \cmd_1\ \ldots\ \bot)\ \thread_2\ \ldots)\ ctp_s\ \epsnd\ \eprcv\ \epwait\ \epbarrier\ \rcvp^\prime)
}
\and
\inferrule[Wait (rcvi) Command 3]
{
  \epwait(\aid_w) = ([\aid_2\ v_{frm2}\ v_{to2}]\ [\aid_3\ v_{frm3}\ v_{to3}]\ \ldots\ [\aid_1\ v_{frm1}\ v_{to1}]) \\
  \aid_2 \notin \rcvp \\ \eprcv(v_{to2},v_{frm2}) < \epsnd(v_{to2},v_{frm2}) \\ \eprcv(v_{to2},\ast) < \epsnd(v_{to2},\ast) \\
   \eprcv(v_{to2},v_{frm2}) = v_c \\
    \eprcv^\prime = \eprcv [(v_{to2}, v_{frm2}) \mapsto v_c + 1]] \\ 
    \epwait^\prime = \epwait [\aid_w\ \mapsto\ ([\aid_3\ v_{frm3}\ v_{to3}]\ \ldots\ [\aid_1\ v_{frm1}\ v_{to1}])]
}{
  ((\thread_0\ \ldots\ ((\wait\ \aid_w\ \aid_1)\ \cmd_1\ \ldots\ \bot)\ \thread_2\ \ldots)\ ctp_s\ \epsnd\ \eprcv\ \epwait\ \epbarrier\ \rcvp)\\ \reduce{m}
  ((\thread_0\ \ldots\ ((\wait\ \aid_w\ \aid_1)\ \cmd_1\ \ldots\ \bot)\ \thread_2\ \ldots)\ ctp_s\ \epsnd\ \eprcv^\prime\ \epwait^\prime\ \epbarrier\ \rcvp)
}
\and
\inferrule[Barrier Command 1]
{
  \epbarrier(\aid_0) = v_c \\ v_c < \npro \\ \epbarrier^\prime = \epbarrier[\aid_0 \mapsto  v_c + 1]
}{
 ((\thread_0\ \ldots\ ((\barrier\ \num_0\ \aid_0)\ \cmd_1\ \ldots\ \bot)\ \thread_2\ \ldots)\ ctp_s\ \epsnd\ \eprcv\ \epwait\ \epbarrier\ \rcvp)\\ \reduce{m}
 ((\thread_0\ \ldots\ ((\barrier\ \num_0\ \aid_0)\ \cmd_1\ \ldots\ \bot)\ \thread_2\ \ldots)\ ctp_s\ \epsnd\ \eprcv\ \epwait\ \epbarrier^\prime\ \rcvp)
}
\and
\inferrule[Barrier Command 2]
{
  \epbarrier(\aid_0) = \npro
}{
 ((\thread_0\ \ldots\ ((\barrier\ \num_0\ \aid_0)\ \cmd_1\ \ldots\ \bot)\ \thread_2\ \ldots)\ (\thread_0^\prime\ \ldots\ (\cmd_2\ \ldots\ \bot)\ \thread_2^\prime\ \ldots)\ \epsnd\ \eprcv\ \epwait\ \epbarrier\ \rcvp)\\ \reduce{m}
 ((\thread_0\ \ldots\ (\cmd_1\ \ldots\ \bot)\ \thread_2\ \ldots)\ (\thread_0^\prime\ \ldots\ (\cmd_2\ \ldots\ (\barrier\ \num_0\ \aid_0)\ \bot)\ \thread_2^\prime\ \ldots)\ \epsnd\ \eprcv\ \epwait\ \epbarrier\ \rcvp)
}

\end{mathpar}}
\caption{Machine Reductions ($\reduce{m}$). }
\label{fig:machine}
\end{figure*}


The function $\mathrm{SCHEDULEFINDER}$ in \algoref{algo:main} is presented as set of rewriting commands for the machine state (\textit{st}) defined in \figref{fig:machine}. These commands define how to execute the program $\mathit{ctp}$ and generates a schedule $\mathit{ctp}_s$. The \emph{Sndi Command} in \figref{fig:machine} consumes the first send in any process of the program. 
%Consider a portion of the \emph{Sndi Command}:
%\begin{eqnarray*}
% \epsnd(v_{to},v_{frm}) = v_c\\ \epsnd^\prime = \epsnd[(v_{to},v_{frm}) \mapsto v_c +1] \\ \epsnd(v_{to},\ast) = v_i \\ \epsnd^{\prime\prime} = \epsnd^\prime[(v_{to},\ast) \mapsto v_i +1] 
%\end{eqnarray*}
$\epsnd^\prime$\ is a new set, just like the old set $\epsnd$, only the new set maps the destination
endpoint $v_{to}$\ and the source endpoint $v_{frm}$ to the number $v_c + 1$ where $v_c$ is the content in the old set. The set is also updated such that it maps $v_{to}$\ and $\ast$ indicating any source to the number $v_i + 1$ where $v_i$ is the content in the old set. The send is then pushed to the schedule $\mathit{ctp}_s$ at the bottom of the corresponding process in the machine state. The \emph{Rcvi Command} in \figref{fig:machine} consumes the receive by updating the set \epwait. Similar to the rule in \emph{Sndi Command}, \epwait\ merely adds a new record for the receive that is indexed by its nearest-enclosing wait. The receive is then moved to the schedule. The \emph{Wait (Rcvi) Command} operates in three ways. If the wait $\aid_w$ maps to an empty set in \epwait\ indicating that no receives need to be completed by $\aid_w$, then the command consumes $\aid_w$ by moving it to the schedule. If the first receive $\aid_2$ in $\epwait(\aid_w)$ is stored in \rcvp\ for the pattern instance $\mathit{pt}$, then $\aid_2$ is removed from \rcvp\ indicating that it is witnessed. As a note, \algoref{algo:main} checks if all the receives in \rcvp\ are witnessed after detecting a schedule for $\mathit{pt}$. The last way that a wait can move forward checks whether the first receive $\aid_2$\ in $\epwait(x_w)$ is able to be consumed. Note that the receives in $\epwait(\aid_0)$ are ordered as they are on the original process of the program. Therefore, if the first receive in $\epwait(x_w)$ cannot be consumed, the following receives are blocked as well. This step requires two conditions. First, $\aid_2$ is not a receive in the pattern instance. Second, there are more sends than receives with common source and destination endpoints and there are more sends for the preceding wildcard receives. If both conditions are satisfied, then the set \eprcv\ is updated where the new set maps the destination endpoint $v_{to}$\ and the source endpoint $v_{frm}$ to the number $v_c + 1$ where $v_c$ is the content in the old set. This receive is then removed from \epwait. The \emph{Barrier Command} moves the barrier forward by its synchronization rule. If the count of the witnessed barriers $\epbarrier(\aid_0)$ for a specific communicator $\aid_0$ is less than the size of processes $N_{pro}$ indicating that the barriers for $\aid_0$ are not matched, then the barrier is not consumed and $\epbarrier(\aid_0)$ is incremented. The barrier can only be moved to the schedule if the count $\epbarrier(\aid_0)$ is equal to the number of processes $N_{pro}$.

The machine rewrites the state \textit{st} until no more reduction rules can be applied indicating that there is no way to further traverse the program. In such a schedule, the last statement on any process is either the bottom of the process or a blocking operation. A blocking operation could be a wait or a barrier. 

\subsection{Correctness}

\algoref{algo:main} is sound meaning any reported deadlock in \algoref{algo:main} implies a real deadlock in program execution. The soundness is formally established by \lemmaref{lemma:mismatch} and \lemmaref{lemma:circular}.

\begin{lemma}[Soundness for Mismatched Send and Receive Pattern]
For any message passing program, \textit{ctp}, a satisfying assignment returned by the function $\mathrm{ENCODE}$ in \algoref{algo:main} indicates a real deadlock schedule for an instance of the mismatched send and receive pattern, $\mathit{pt}$. 
\label{lemma:mismatch}
\end{lemma}
\begin{proof}
Proof by showing the existence of a deadlock. First, assume $(\mathit{ctp}^\prime,$ $\mathit{ctp}_s,$ $\mathit{N_s},$ $\mathit{N_r},$ $\mathit{P_r},$ $\mathit{P_b},$ $\mathit{R_{pt}}^\prime)$ is the machine state after applying the operational semantics in \figref{fig:machine} and no reduction commands can be further applied. Second, assume the schedule $\mathit{ctp}_s$ is input to the function $\mathrm{ENCODE}$ in \algoref{algo:main}. Since $\mathrm{ENCODE}$ returns a satisfying assignment, say $\mathit{t}$, it implies that $\mathit{t}$ is a sequential order of all the operations in $\mathit{ctp}_s$ that is executable by the runtime system. Also, for the process that the deterministic receive $\mathtt{r_c}$ in $\mathit{pt}$ resides in, say $\mathit{p}$, $\mathit{t}$ satisfies that each send with the destination $\mathit{p}$ is matched with some receive in $\mathit{ctp}_s$. As such, no sends can match $\mathtt{r_c}$ as the nearest-enclosing wait for $\mathtt{r_c}$ is stored in $\mathit{ctp}^\prime$ according to the \emph{Wait (Rcvi) Command 2} in \figref{fig:machine}. Further, for any process other than $\mathit{p}$ in $\mathit{pt}$, the first operation is blocked in $\mathit{ctp}^\prime$ according to the operational semantics in \figref{fig:machine}. As such, $\mathit{ctp}^\prime$ deadlocks. Since $\mathit{ctp}_s$ is executable and $\mathit{ctp}^\prime$ deadlocks, the program \textit{ctp} may deadlock for the pattern instance $\mathit{pt}$ in program execution. 
\end{proof}

\begin{lemma}[Soundness for Circular Dependency Pattern]
For any message passing program, \textit{ctp}, the existence of a deadlock reported by \algoref{algo:main} for an instance of the circular dependency pattern, $\mathit{pt}$, indicates a real deadlock. 
\label{lemma:circular}
\end{lemma}
\begin{proof}
Proof by showing the existence of a deadlock. First, assume $(\mathit{ctp}^\prime,$ $\mathit{ctp}_s,$ $\mathit{N_s},$ $\mathit{N_r},$ $\mathit{P_r},$ $\mathit{P_b},$ $\mathit{R_{pt}}^\prime)$ is the machine state after applying the operational semantics in \figref{fig:machine} and no reduction commands can be further applied. Second, assume \algoref{algo:main} reports the existence of a deadlock at line $15$. Assume $\mathit{ctp}_s$ is a feasible schedule. Similar to the proof of \lemmaref{lemma:mismatch}, $\mathit{ctp}_s$ is executable in the runtime system. Also, any receive in $\mathit{pt}$ has no way to be matched if the condition at line $11$ in \algoref{algo:main} is not satisfied for any receive in $\mathit{pt}$. Further, for any process other than the processes in $\mathit{pt}$, the first operation is blocked in $\mathit{ctp}^\prime$ according to the operational semantics in \figref{fig:machine}. As such, $\mathit{ctp}^\prime$ deadlocks. As such, the program \textit{ctp} may deadlock for the pattern instance $\mathit{pt}$ in program execution. 
\end{proof}

\algoref{algo:main} is also complete for the two deadlock patterns. 

\begin{lemma}[Completeness for Two Common Deadlock Patterns]
For any message passing program, \textit{ctp}, any real deadlock for mismatched send and receive pattern and circular dependency pattern can be reported by \algoref{algo:main}.
\label{lemma:complete}
\end{lemma}
\begin{proof}
Proof by showing that the operational semantics in \figref{fig:machine} simulate the message communication under infinite buffer semantics. For \emph{Sndi Command} and \emph{Rcvi Command}, a send or receive is consumed immediately by incrementing two structures $\mathit{N_s}$ and $\mathit{N_r}$, respectively. This is consistent with the issuing of send and receive under infinite buffer semantics. The three cases of \emph{Wait Command} witness the completion of receives that are not in the pattern $\mathit{pt}$ and hold the receives in $\mathit{pt}$ so that a postponed feasibility check is launched in \algoref{algo:main}. The two cases of \emph{Barrier Command} stops the execution of a member process until all the barriers in group are witnessed. Since the operational semantics in \figref{fig:machine} are able to simulate the behavior under infinite buffer semantics, if a real deadlock exists for a pattern, then the schedule is able to be extracted. Thus, the deadlock can be reported.
\end{proof}

As a note, \algoref{algo:main} is not complete because there might exist other deadlock patterns that cause a message passing program deadlock. Therefore, a reported deadlock free program may have a deadlock for other patterns. The completeness, however, can be proved if all the deadlock patterns are extended to \algoref{algo:main}. 

 
 %The completeness is established by \lemmaref{lemma:completeness}.
 
 %\begin{lemma}[Completeness]
 %For any message passing program, \textit{ctp}, that may deadlock for mismatched send-receive pattern or circular dependency pattern, \algoref{algo:main} reports a deadlock.
% \label{lemma:completeness}
 %\end{lemma}
 %\begin{proof}
 %Proof by two steps. The first step proves that all the pattern instances are detected. For the mismatched send-receive pattern, since the algorithm traverses every process and detects all the deterministic receives that satisfy the properties of \defref{def:mismatch}, all the pattern instances are detected. For the circular dependency pattern, the algorithm adds all the receives and sends that may be contained in sequential relations defined in \defref{def:seqrelation} to the vertices $\mathit{V}$. It also adds the over-approximated match relations and all possible happens-before relations that satisfy the properties of \defref{def:circular} to the edges $\mathit{E}$. Further, the Johnson's algorithm is able to detect all the cycles given $\mathit{V}$ and $\mathit{E}$. As such, all the pattern instances are detected.
%The second step proves that if there are a set of pattern instances, \textit{PT}, for \textit{ctp} such that each instance may cause a real deadlock, at least one deadlock is detected. There are two cases for any pattern instance $\mathit{pt}\in\mathit{PT}$. First, if a deadlock is reported for $\mathit{pt}$, the statement above holds. Second, no deadlock is reported for $\mathit{pt}$ but there is a real deadlock for it. This is because either no schedule is detected but one actually exists, or the schedule detected is infeasible for the mismatched send-receive pattern. Under either situation, a ``bogus" schedule is detected but is pruned by \algoref{algo:main}. Notice that the operational semantics in \figref{fig:machine} determines if a receive can be matched with the rule of ``\emph{Wait (Rcvi) Command 3}" in \figref{fig:machine}. For any wildcard receive $\mathtt{r_w}$, if the condition holds, then $\mathtt{r_w}$ can also be matched in runtime. For any deterministic receive $\mathtt{r_c}$, if this condition holds, $\mathtt{r_c}$ may not be matched because the schedule has to match all the available sends with the preceding wildcard receives on a single process. If this happens, a ``bogus" schedule exists with a real deadlock for the mismatched send-receive pattern instance $\{\mathtt{r_c}\}$. Since all the pattern instances are detected, the pattern instance $\{\mathtt{r_c}\}$ is eventually checked. There are only two cases for any pattern instance. It is not possible that each pattern instance has a real deadlock in its ``bogus" schedule. As such, at least one deadlock is detected for some pattern instance. Therefore, the statement above holds.
%Since all the pattern instances are detected and at least one deadlock is detected if real deadlocks exists for those pattern instances, the completeness is proved. $\Box$
%\end{proof}

\subsection{Deadlock for Zero Buffer Semantics}
The deadlock patterns discussed above may also cause a program to deadlock under zero buffer semantics. Notice that the commands in \figref{fig:machine} are consistent with how messages communicate under infinite buffer semantics (e.g., a send is consumed immediately). The zero buffer semantics, however, enforce a different way of message communication. As such, a feasible infinite buffer schedule generated by \algoref{algo:main} is not able to witness a deadlock under zero buffer semantics. Therefore, the zero buffer compatibility should be checked for each generated schedule. \algoref{algo:main} is extended with two changes: 1) the zero buffer encoding rules (refer to \cite{HuangNFM15}) are added to the function $\mathrm{ENCODE}$; 2) The circular dependency pattern also needs an SMT encoding to check the feasibility of a schedule. As a note, to encode the match pairs for the circular dependency pattern only needs to ensure that each receive in the schedule is matched with some send. Once the schedule is proved to be zero buffer compatible, the program may deadlock under zero buffer semantics; otherwise, the pattern instance does not imply a real deadlock. 

%\examplefigfive

%Notice that the zero buffer semantics may cause a deadlock in a different way other than the patterns discussed above. For instance, the program in \figref{fig:zeropattern} deadlocks under zero buffer semantics if the send at line \texttt{20} matches the receive at line \texttt{00}. There is no way to match the send at line \texttt{02} or the send at line \texttt{10}. This program does not contain either pattern discussed in this paper. To check this type of deadlock needs to define deadlock patterns merely for zero buffer semantics. These patterns, however, are hard to be completely defined because all sends are blocking operations under zero buffer semantics. A deadlock may randomly occur in the way of message communication and a new pattern can be formed. Future work explores to detect this type of deadlock.



\section{Experiments}
The experiments compare the performance of the approach in this paper with two state-of-art MPI verifiers including MOPPER \cite{DBLP:conf/fm/ForejtKNS14}, a SAT based tool, and ISP \cite{DBLP:conf/ppopp/VakkalankaSGK08,DBLP:conf/sbmf/SharmaGB12}, a dynamic analyzer. 

MOPPER is designed to verify only single-path programs. To compare with ISP, it is meaningful only when the benchmarks are single-path.   
Therefore, a series of experiments are conducted for a set of single-path programs, including three small programs where each contains a deadlock, and seven typical benchmark programs. All the results show the comparison under infinite buffer semantics.  
%The computation is not included in each program because it is irrelevant to the problem of deadlock. 
The experiments are run on a AMD A8 Quad Core processor with 6 GB of memory running Ubuntu 14.04 LTS. We set a time limit of 30 minutes for each test. We abort the verification process if it does not complete within the time limit. 


\begin{savenotes}
\begin{table*}[t]
\begin{center}
\small
\caption{Tests on Selected Benchmarks}\label{table:benchmarks}
     \begin{threeparttable}
\begin{tabular}{|c|c|c|c|c|c|c|c|c|c|c|c|c|}
		\hline
         \multicolumn{5}{|c|}{Test Programs} & \multicolumn{4}{c|}{Our Method} & \multicolumn{2}{c|}{ISP} & \multicolumn{2}{c|}{MOPPER}  \\ \hline
          $Name$ & \#Procs & \#Calls&Match&Deadlock &PT & PTR &D & Time & D &Time & D & Time\\ \hline
           \textit{dlg1} & 3 & 8 & 2  & Yes & 1& 0&$\surd$ & 0.009s & $\surd$ & 0.116s & $\surd$  & 0.001s\\ \hline
          \hline
           \textit{dlg5} & 3 & 16 & 12  & Yes &1&0 & $\surd$ & 0.021s & $\surd$ & 0.118s & $\surd$ & 0.002s\\ \hline
          \hline
           \textit{dlg8} & 3 & 12 & 4 & Yes &1& 0& $\surd$ & 0.019s & $\surd$ & 0.110s & $\surd$ & 0.002s\\ \hline
          \hline
          %\textit{Mismatch} & 3 & 12 & 2  & Yes  &1& & $\surd$ & 0.018s &  $\surd$ & 0.091s & $\surd$ & 0.002s\\ \hline
          %\hline
          
          %\textit{Circular} & 3 & 16 & 1  & Yes  &1& & $\surd$ & 0.004s & $\surd$ & 0.057s &  $\surd$ & 0.001s\\ \hline
          %\hline
          
          \multirow{3}{*}{\textit{Monte}} & 4 & 35 &  24 
          												     %& 0 & No & No\tnote{\textdagger} & 3.62 & 0.02s & 6 & 0.25s & 6.09 & $<$0.01s\\ \cline{5-13}
          						         & No  &0 &0 &  & 0.001s &  & 0.957s &  & 0.015s\\ \cline{2-13}
						       		& 8 & 75 &  40K 
          												     %& 0 & No & No\tnote{\textdagger} & 4.83 & 0.04s & $>$5K & TO & 11.28 & 0.02s\\ \cline{5-13}
          						        & No  &0&0 &  & 0.002s &  & TO &  & 0.751s\\ \cline{2-13}
						              & 16 & 155 &  2E13 
          												     %& 0 & No & No\tnote{\textdagger} & 8.97 & 0.29s & $>$5K & TO & 24.42 & 0.08s\\ \cline{5-13}
          						        & No  &0& 0& & 0.006s &  & TO &  & TO\\ \hline
						       \hline
						       
	   \multirow{3}{*}{\textit{Integrate}} & 8 & 36 &  5K
          												     %& 0 & Yes & No & 4.71 & 0.08s & 1 & 0.15s & --  & -- \tnote{a}\\ \cline{5-13}
          						        & No  &0 & 0&  & 0.001s &  & $>$1000s &  & 0.103s\\ \cline{2-13}
						       		& 10 & 46 & 362K
          												     %& 0 & Yes & No & 5.39 & 0.08s & 1 & 0.16s & -- & -- \tnote{a}\\ \cline{5-13}
          						        & No  &0&0 &  & 0.002s & & TO &  & 34.986s\\ \cline{2-13}
						              & 16 & 76 &  1E12
          												   %  & 0 & Yes & No & 8.79 & 0.62s & 1 & 0.25s & -- & -- \tnote{a}\\ \cline{5-13}
          						       & No  &0&0 & & 0.003s &  & TO & & TO\\ \hline
						       \hline
						       
	 %$\mathit{Integrate_m}$ & 10 & 46 & 362K
          												     %& 0 & Yes & No & 5.39 & 0.08s & 1 & 0.16s & -- & -- \tnote{a}\\ \cline{5-13}
          %						        & Yes  &1& $\surd$ & 0.024s & & TO & $\surd$ & 0.091s\\ \hline
		%					\hline
						       
	    \multirow{2}{*}{\textit{Diffusion2D}} & 4 & 52 & 6E9 
          												    % & 0 & No & Yes & 5.50 & 0.04s & 90 & 3.09s & 6.10 & 0.01s\\ \cline{5-13}
          						        & No  &0&0 &  & 0.003s &  & 32.005s &   & 0.039s\\ \cline{2-13}
						       		& 8 & 108 & 2E21 
          												     %& 0 & No & Yes & 11.94 & 0.22s & $>$9K & TO & -- & TO\\ \cline{5-13}
          						         & No  &0 &0 &  & 0.004s &  & TO &  & TO\\ \hline
						       \hline
						       
	    %  $\mathit{Diffusion2D_m}$ & 8 & 108 & 2E21 
          												     %& 0 & No & Yes & 11.94 & 0.22s & $>$9K & TO & -- & TO\\ \cline{5-13}
          	%					         & Yes  &1& $\surd$ & 0.036s &  & TO & $\surd$ & 0.362s\\ \hline        
		%				       \hline

						       
	    \multirow{2}{*}{\textit{Floyd}} & 8 & 120 &  4E29 
          												     %& 0 & No & No & 13.87 & 0.15s & $>$20K & TO & 18.05 & 0.27s\\ \cline{5-13}
          						        & No  &0 &0&  & 0.004s &  & TO &  & 2.812s\\ \cline{2-13}
						       		& 16 & 256 &  1E58 
          												    % & 0 & No & No & 21.58 & 0.26s & $>$20K & TO & 67.53 & 43.08s\\ \cline{5-13}
          						         & No  &0 & 0&  & 0.006s &  & TO &  & 62.467s\\ \hline
						        \hline
	    \multirow{2}{*}{\textit{GE}} & 8 & 56 & 64  
          												     %& 0 & No & No & 13.87 & 0.15s & $>$20K & TO & 18.05 & 0.27s\\ \cline{5-13}
          						        & No  &0 &0&  & 0.011s &  & 1.054s &  & 0.042s\\ \cline{2-13}
						       		& 16 & 120 & 16K  
          												    % & 0 & No & No & 21.58 & 0.26s & $>$20K & TO & 67.53 & 43.08s\\ \cline{5-13}
          						         & No  &0 & 0&  & 0.014s &  & 1.426s &  & 0.098s\\ \hline
						        \hline
			        
             
						        
	     \multirow{2}{*}{\textit{Mismatch}} & 3 & 400 & 100  & Yes  &50 &49 & $\surd$ & 1.609s  & $\surd$  & 4.274s & $\surd$ & 2.601s\\ \cline{2-13}
          					& 3 & 800 & 2E40  & Yes  &100& 98 & $\surd$ & 11.027s  & $\surd$ & 514.852s & $\surd$ & 17.892s\\ \hline
          \hline
          
           \textit{Circular} & 3 & 252 & 8E257  & Yes  & 132K & $\sim$132K & $\surd$ & 13.821s  &  & TO & $\surd$ & 728.722s\\ \hline

						                       
\end{tabular}
\end{threeparttable}
\end{center}
\end{table*}
\end{savenotes}

The results of the comparison are in \tableref{table:benchmarks}. The column ``Match" records the approximated number of match possibilities. A program with a large number of match possibilities has a large degree of message non-determinism. The column ``Deadlock" indicates the existence of deadlocks. The column ``PT" is the number of pattern instances that the algorithm in this paper detects. The column ``PTR" is the number of pattern instances that are pruned by the algorithm in this paper. The column ``D" indicates whether the tool detects a deadlock or not. The ``Time" column for our approach is the time of static analysis and constraint solving (if necessary). The ``Time" column of MOPPER is for constraint generation and solving. The column ``Time" for ISP is the running time of dynamic analysis. 

%The meaning of the symbol ``--" is ``unavailable": either the test is not interesting for comparison or the error is detected in preprocessing.
 
%\begin{compactitem}

\textit{dlg1, dlg5 and dlg8} implement simple message communication \cite{DBLP:conf/fm/ForejtKNS14}. Each contains a deadlock for mismatched send and receive pattern. 

\textit{Monte} implements the Monte Carlo method to compute $\pi$ \cite{benchmark:mentoCarlo}. It uses one manger process and multiple worker processes to send messages back and forth. In addition, barrier operations are used to synchronize the program. 

\textit{Integrate} uses heavy non-determinism in message communication to compute an integral of the $\sin$ function over the interval $[0, \pi]$ \cite{benchmark:fevs}. This benchmark also has a manger-worker pattern where the root process divides the interval to a certain number of tasks. It then distributes those tasks to multiple worker processes. 

%$\mathit{Integrate_m}$ is a variation of \textit{Integrate} such that it now has one instance of mismatched send-receive pattern. 
 
\textit{Diffusion2D} has an interesting computation pattern that uses barriers to ``partition" the message communication into several sections \cite{benchmark:fevs}. A message from a send can be only received in a common section. 

%$\mathit{Diffusion2D_m}$ is a variation of \textit{Diffusion2D}. It contains one instance of mismatched send-receive pattern.

\textit{Floyd} implements the shortest path algorithm for all the pairs of nodes \cite{DBLP:conf/ppopp/XueLWGCZZV09}. Each node communicates only with the immediate following neighbor.

\textit{GE} is a message passing implementation for Gaussian Elimination  \cite{DBLP:conf/ppopp/XueLWGCZZV09}. Messages are communicated by issuing several wildcard receives on each node. 

\textit{Mismatch} implements the message communication that contains a set of mismatched send and receive pattern. 

\textit{Circular} implements the message communication that contains a deadlock for circular dependency pattern.

%\end{compactitem}

The results show that the algorithm in this paper is highly efficient compared to ISP and MOPPER. For the small programs (e.g., \textit{dlg1}), all the tools correctly find the deadlocks and return very fast. For the large benchmark programs that do not have any deadlock (e.g., \textit{monte}), ISP runs much slower than the approach in this paper and MOPPER. For instance, it runs out of time when the program size of \textit{monte} increases to 8 processes. MOPPER runs fast when the number of processes is small, however, its runtime is largely increased as the number of processes increases. The approach in this paper, however, returns very fast even if the number of processes is large. This is because the approach in this paper does not need to run the SMT encoder as the algorithm detects no deadlock pattern instances. To test the efficiency of the schedule finder and the SMT encoder in this paper, the experiments include two large programs $\mathit{Mismatch}$ and $\mathit{Circular}$ that contain deadlocks for two patterns discussed in this paper. As shown in the results, the approach in this paper is able to prune massive number of infeasible deadlock patterns. For example, the program $\mathit{Circular}$ prunes about 132K infeasible pattern instances and detects a real deadlock. The results show that the algorithm in this paper is faster than ISP and MOPPER for the programs $\mathit{Mismatch}$ and $\mathit{Circular}$. As discussed earlier, our method may not find deadlocks that do not fit the patterns of mismatched send-receive or circular dependency. However, the experimental results show that our method is able to find all the deadlocks in the benchmark programs. This demonstrates that our method has a high coverage of deadlock detection. 


\section{Related Work}
The approach in this presentation is inspired by several works. The predictive analysis collects a single trace and predicts deadlocks in the other traces with the same input  \cite{DBLP:conf/sc/SharmaGB12,Subodh:Dissertation}. The dependency constructor in the work refines the match pairs that may lead to a deadlock. The refining strategy uses simple counting rules that inspires the abstract machine in this presentation. However, the approach in this presentation does not refine the match pairs for deadlock detection.

Joshi et al. proposed a method that finds real deadlocks for multi-threaded Java programs by first detecting potential lock dependency cycles with a imprecise dynamic analyzer and then finding real deadlocks by a random thread scheduler with high probability \cite{DBLP:conf/pldi/JoshiPSN09}. The solution scales to large programs. Also, the method detects a number of previously unknown deadlocks in a set of benchmarks. The refining strategy of the work also inspires the general algorithm in this presentation, only this presentation intends to prune a set of potential deadlock instances instead of finding deadlocks from an instance.

A precise SMT encoding technique is proposed for detecting user-provided assertions for MCAPI programs \cite{DBLP:conf/kbse/HuangMM13}. The encoding is sound and complete and is easy to use to reason about infinite buffer semantics without requiring a precise match set. The work also provides an algorithm that runs in quadratic time complexity to generate a sufficiently small over-approximated match set based on the given execution trace. This approach is extended to checking zero buffer incompatibility for MPI semantics \cite{HuangNFM15}. 
%The extended approach provides a set of simple rules for encoding zero buffer semantics. 
The technique is also used for deadlock detection for MPI programs in this presentation.

%Comparing the approach in this paper with three works above \cite{DBLP:conf/sc/SharmaGB12,Subodh:Dissertation,DBLP:conf/pldi/JoshiPSN09,DBLP:conf/kbse/HuangMM13}. The main framework of the algorithm in this paper is inspired by the refining strategy of \cite{DBLP:conf/pldi/JoshiPSN09} that creates deadlocks from lock dependency cycles, only this paper intends to prune the set of potential deadlock instances for deadlock detection. This paper is also inspired by the counting rules in the predictive analysis \cite{DBLP:conf/sc/SharmaGB12,Subodh:Dissertation}, but these rules are not used to refine the match pairs in this paper. Further, the SMT encoding used in this paper is relied on the work for MCAPI verification \cite{DBLP:conf/kbse/HuangMM13}. However, it is applied to MPI verification in this paper.

There are other solutions for message passing program analysis.
The dynamic analyzer ISP implements the POE algorithm, a Dynamic Partial Order Reduction (DPOR) algorithm \cite{DBLP:conf/popl/FlanaganG05} applied to MPI programs \cite{DBLP:conf/ppopp/VakkalankaSGK08}. An extension is the MSPOE algorithm \cite{DBLP:conf/sbmf/SharmaGB12}. It operates by postponing the cooperative operations for message passing in transit until each process reaches a blocking call. It then determines the potential matches of send and receive operations in the runtime. The solution is able to detect errors such as assertion violation and deadlock in an MPI program.

Forejt et al. proposed a SAT based approach to detect deadlock in a single-path MPI program \cite{DBLP:conf/fm/ForejtKNS14}. The solution is correct and efficient for programs with a low degree of message non-determinism. However, since the size of their encoding is cubic, checking large programs is time consuming. Similar to the solution in this presentation, the work requires a match pair set that can be over-approximated.

Umpire is an approach of runtime verification for checking multiple MPI errors such as deadlock and resource tracking \cite{DBLP:conf/sc/VetterS00}. The error checking is taken by spawning one manger thread and several outfielder threads in the execution of an MPI program. The tool fails for two reasons. First, the deadlock detection only considers dependency cycles, therefore, is not able to detect deadlock for orphaned receive pattern. Second, it does not scale well since the asynchronous trace transfer highly depends on the thread interleavings which are exponential. An extension to Umpire is Marmot \cite{DBLP:conf/parco/KrammerBMR03}. The work uses a centralized sever instead of multiple threads for error checking. Marmot is able to check local errors. However, it suffers from the same disadvantage of Umpire. Another extension to Umpire is MUST \cite{DBLP:conf/ptw/HilbrichSSM09}. The structure of MUST allows the users to execute the error checking either in an application process itself or in extra processes that are used to offload these analyses. Therefore, MUST scales to programs with more than 1,000 processes. However, just like Umpire and Marmot, the approach is neither sound nor complete for deadlock detection. 


MPI-Spin is integrated in the model checker SPIN \cite{DBLP:journals/tse/Holzmann97}, for verifying MPI programs \cite{DBLP:conf/vmcai/Siegel07,DBLP:conf/pvm/Siegel07}. It generates a model of an MPI program and symbolically executes it. It does not scale to large programs with a large degree of message non-determinism.

Vo et al. used Lamport clocks to update the auxiliary information via piggyback messages \cite{DBLP:conf/sc/VoAGSSB10,DBLP:conf/IEEEpact/VoGKSSB11}. While completeness is abandoned in their analysis, they show the work is useful and efficient in practice. 

Sharma et al. proposed the first push button model checker for MCAPI -- MCC \cite{DBLP:conf/fmcad/SharmaGMH09}. It indirectly controls the MCAPI runtime to verify MCAPI programs under zero buffer semantics. An obvious drawback of the work is its inability to analyze infinite buffer semantics which is known as a common runtime environment in message passing. A key insight, though, is the direct use of match pairs -- couplings for potential sends and receives.

Elwakil et al. also used SMT techniques to reason about the program behavior in the MCAPI domain \cite{DBLP:conf/issta/ElwakilY10,DBLP:conf/atva/ElwakilYW10}. State-based and order-based encoding techniques are both used. These techniques fail to reason about the infinite buffer semantics and require a precise match set which is non-trivial to compute beforehand.


\section{Conclusion and Future Work}
This presentation presents a new algorithm that first detects the potential deadlock pattern instances and then detects a real deadlock by pruning from these instances.
The key insight in this presentation is the abstract machine that prunes provably non-feasible schedules by simply counting the issued sends and issued receives. 
This presentation further defines two types of deadlock patterns: circular dependency and orphaned receive, and their validation. The circular dependency uses simple rules for validation. The orphaned receive, however, requires a higher cost SMT encoding from existing work. This presentation additionally proves that the algorithm is sound and complete for the circular dependency pattern and the orphaned receive pattern.
Further, the algorithm can be modified to check deadlocks for these two patterns under zero buffer semantics. Experiments demonstrate that the algorithm in this presentation is able to detect all the deadlocks in a typical set of benchmarks and is more efficient than two state-of-art MPI verifiers.

It is believed that there exist deadlock patterns other than the circular dependency and the orphaned receive in MPI programs.
Future work will define other deadlock patterns and to detect deadlocks for these patterns. Also, future work will explore how to detect deadlocks for programs with branching. 

\newpage

\newpage
\bibliographystyle{abbrvnat}
\bibliography{bib/paper}
\end{document}