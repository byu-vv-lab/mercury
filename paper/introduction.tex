\section{Introduction}
%The Message Passing Interface (MPI) is widely used in high performance computing (HPC). A common problem in any MPI program, is communication deadlock. A communication deadlock is ``\emph{a situation in which each member process of the group is waiting for some member process to communicate with it, but no member is attempting to communicate with it}" \cite{DBLP:conf/fsttcs/Natarajan84}. This paper refers to a communication deadlock as deadlock. 

The Message Passing Interface (MPI) is widely used in high performance computing (HPC). A common problem in any MPI program is deadlock. The deadlock problem for a single-path MPI program is NP--Complete \cite{DBLP:conf/fm/ForejtKNS14}.
This paper considers only single-path MPI programs that are typical of many HPC applications and the standard for most deadlock analysis tools.  
The deadlock in a single-path MPI program is difficult to detect because of several reasons. First, the message race, intended or not, leads to message non-determinism such that a receive may be matched with more than one send by the the runtime. This non-determinism in MPI programs is difficult to test or debug. Second, the collective operations such as barrier synchronize a program making the semantics more complicated. Finally, the message communication is impacted by two buffering semantics: infinite buffer semantics (messages are buffered in the system) and zero buffer semantics (no buffering in the system). As a note, the algorithm in this paper is designed for infinite buffer semantics. Nevertheless, it is adaptable to zero buffer semantics with a few changes as discussed in Section 4 and Section 5.

There are several solutions proposed for detecting deadlocks in MPI programs. However, these solutions do not scale well for large programs. Among these solutions, dynamic analysis include the POE algorithm that is capable of analyzing the behavior of MPI programs \cite{DBLP:conf/ppopp/VakkalankaSGK08}. This algorithm is implemented by a modern MPI verifier, ISP. Also, an extension is the MSPOE algorithm is more efficient for deadlock detection\cite{DBLP:conf/sbmf/SharmaGB12}. 
%The problem can also be checked by SAT/SMT technique. 
The SAT/SMT technique can also be applied to check deadlocks in MPI programs. 
Forejt et al. proposed a SAT based approach to detect deadlock in a single-path MPI program \cite{DBLP:conf/fm/ForejtKNS14}. This approach is much faster than the POE algorithm and the MSPOE algorithm, but it also suffers from the scalability problem for large MPI programs. 
As far as we know, these three approaches are the only existing works in MPI deadlock detection. 
To make the deadlock detection scalable, this paper is inspired by an interesting work in the field of shared memory programs.
This work uses a random thread scheduler to create the potential deadlocks that are detected by an imprecise dynamic analyzer \cite{DBLP:conf/pldi/JoshiPSN09}. This approach is able to check large multi-threaded programs and does not give any false warnings. 

This paper presents a new algorithm that is able to detect a deadlock for a single-path MPI program in three steps. First, the algorithm uses a static analyzer to detect a set of pattern instances by statically traversing the program. The algorithm then uses an abstract machine to prune provably non-feasible pattern instances from the set of potential deadlocks. If needed, the algorithm finally validates whether or not any remaining instances imply a real deadlock. Novel is that the abstract machine efficiently rejects non-feasible instances by simply counting the issued sends and receives, instead of exhaustively enumerating all message races. The complexity of the algorithm is quadratic. 

This paper further defines two distinct patterns of deadlock and their validations: circular dependency and orphaned receive. A circular dependency pattern may cause a program deadlock if there exists a cycle among a group of processes where a receive on each member process waits for the issuing of a send on another member process but never gets a response. An orphaned receive pattern may also cause a program deadlock if there exists a pair of receives, a wildcard receive (a receive that may match a send from any source) and on an identical process a deterministic receive (a receive that only matches a send from a fixed source). The deadlock occurs when the wildcard receive is matched with a send that should match the deterministic receive by the runtime. This paper presents how to detect all the instances for each of the patterns above by statically traversing the program. Further, this paper presents simple rules for validating the instances of the circular dependency pattern. For the orphaned receive pattern, the validation requires a higher cost SMT encoding to identify a real deadlock. 
This paper additionally gives proofs that the general algorithm is sound and complete for the circular dependency pattern and the orphaned receive pattern. 

%As a note, the algorithm is designed for infinite buffer semantics. Nevertheless, it is adaptable to zero buffer semantics with a few changes in deadlock validation.

The contributions are summarized as follows.
\begin{itemize}
\item The key contribution is an abstract machine that is able to efficiently prune non-feasible pattern instances from a set of potential deadlocks without false warnings; 
\item The second contribution is the definition of two typical patterns: circular dependency and orphaned receive, and their validations;
\item The third contribution is the implementation of the algorithm in the paper for MPI deadlock detection and a set of benchmarks that demonstrate it is more efficient than two state-of-art MPI verifiers.
\end{itemize}

%%add text such that these two patterns cover most deadlock situations. 

%The rest of the paper is organized as follows: Sections 2 introduces a few notations and definitions for an MPI program; Section 3 presents the main framework of the technique in the paper with a soundness and completeness proof, Section 4 and 5 present the circular dependency pattern and the orphaned receive pattern with a few examples, the pattern matching algorithms and the validation algorithms; Section 6 gives the experimental results; Section 7 discusses the related work; and Section 8 discusses the conclusion and future work.
