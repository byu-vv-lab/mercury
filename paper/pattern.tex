\section{MPI Deadlock Patterns}
This section introduces two patterns that may cause an MPI program deadlock.  The first pattern is the mismatched send-receive. 
Any pattern instance satisfies the following criterial:
\begin{itemize}
\item on a single process there is a wildcard receive followed by a deterministic receive; 
\item at least two sends from different source endpoints may potentially match the wildcard receive; and
\item among these matched sends, at least one send matches the deterministic receive. 
\end{itemize}

\examplefigone

As shown in \figref{fig:deadlock1}, the program has an instance of this pattern where the operations are highlighted using underlines. A deadlock may occur if $\mathtt{S_{1,1}}$ on process $P1$ matches $\mathtt{R_{0,1}}$ on process $P0$ and there is no way to match $\mathtt{S_{2,0}}$ on process $P2$ with $\mathtt{R_{0,2}}$ on process $P0$. A feasible deadlock schedule is $\mathtt{S_{0,0}}$$\mathtt{R_{1,0}}$$\mathtt{S_{1,1}}$$\mathtt{R_{0,1}}$$\mathtt{S_{2,0}}$. Note that the existence of a pattern instance does not mean the existence of a deadlock. The program in \figref{fig:nodeadlock1} does not deadlock even if a pattern instance exists. To be precise, $\mathtt{R_{0,0}}$ on process $P0$ has to match $\mathtt{S_{2,0}}$ on process $P2$, therefore, $\mathtt{R_{0,2}}$ on process $P0$ is able to match $\mathtt{S_{1,1}}$ on process $P1$ eventually. 

\examplefigtwo

The second pattern is the circular dependency for message communication. The circuit in this pattern contains a set of vertices amongst a set of processes. Each vertex is a either a receive or a send. The edge from a receive to a send exists if the receive is followed by the send on a common process. The edge from a send to a receive exists if the receive can potentially match the send. Any pattern instance satisfies the following criterial:
\begin{itemize}
\item at least two processes exist in the pattern instance;
\item for any process in the pattern instance, there exist only one send and only one receive where the receive is followed by the send;
\item for any pair of vertices in the pattern instance, there exists only one path from one vertex to the other; and
\item for any deterministic receive in the patter instance, there are no wildcard receives preceding the deterministic receive on the same process.
\end{itemize}

\figref{fig:deadlock2} shows an MPI program that deadlocks. This deadlock is caused by circular dependency of message communication. Each receive with an underline waits for the issuing of a send from one of the other processes but never gets a response.

\examplefigthree

Similarly, the existence of the circular dependency of message communication does not imply the existence of a deadlock. For instance, the program in \figref{fig:nodeadlock2} does not deadlock because $\mathtt{R_{1,0}}$ on process $P2$ is able to match $\mathtt{S_{0,0}}$ on process $P1$, therefore, the circular dependency does not hold. 

\examplefigfour





