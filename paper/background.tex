\section{Background}
This presentation uses a concurrent trace program (CTP) syntax with semantics loosely based on MPI semantics for convenience. In an MPI program, it is assumed that only non-blocking sends and receives are used. 
A non-blocking operation is issued without blocking the program execution. It is usually paired with a wait function that witnesses the completion of this non-blocking operation.
%% introduce the language aspects: non-blocking, blocking, barrier, collective, point-to-point in the introduction.
In general, the language syntax presented in \figref{fig:expr:stx} (a) is used for the abstract machine discussed in the next section. The presentation uses ellipses ($\ldots$) to represent zero or more repetitions, bold-face to indicate terminals. The language defines a CTP (\textit{ctp}) as a list of processes.  A process (\thread) is a list of commands.  For simplicity, commands (\textit{e}) are non-blocking send (\sendi), non-blocking receive (\recvi), wait (\wait) and barrier (\barrier). Each command is a tuple. A subscript is used to indicate a line number. 
The non-terminal \aid\ in the grammar is a unique string identifier \textbf{ID} associated with a receive or a wait. A receive takes a second identifier belonging to the associated wait. It is assumed that a message is sent out immediately once the non-blocking send is issued, therefore, it is not necessary to associate an identifier or a wait with the send. 
A barrier \barrier\ is also associated with an identifier that is unique for its communicator. A communicator identifies a group of barriers. The barriers block the program execution until all the members in the same group are witnessed.
The non-terminal \num\ in the grammar is a number that represents an endpoint. To be precise, a receive (or a send) is associated with a source endpoint and a destination endpoint. Note that if the source endpoint of a receive is $\ast$, this receive is a wildcard receive meaning it may be matched with a send from any source. The set \rcvp\ is a pattern instance that consists of a set of commands. \figref{fig:deadlock2} shows an example of a CTP with three processes. 
%Consider extending the syntax for data and computation so can use it for other errors
The data and computation are ignored since only message communication is considered in the problem of deadlock. 
\figref{fig:expr:stx} (b) further defines the evaluation syntax that is used to define the semantics in \figref{fig:machine}.


\newsavebox{\boxLangSyntax}

\begin{lrbox}{\boxLangSyntax}
\begin{minipage}[c]{0.3\linewidth}
\cfgstart
\cfgrule{ctp}{\lp\cfgnt{\thread}$~\ldots$\rp}
\cfgrule{\thread}{\lp\cfgnt{\cmd}~$\ldots$~$\bot$\rp}
\cfgrule{\cmd}{\lp\cfgt{\sendi}~\cfgt{\num}~\cfgt{\num}\rp}
   \cfgorline{\lp\cfgt{\recvi}~\cfgt{\aid}~\cfgt{\num}~\cfgt{\num}~\cfgt{\aid}\rp}
   \cfgorline{\lp\cfgt{\wait}~\cfgt{\aid}\rp}
   \cfgorline{\lp\cfgt{\barrier}~\cfgt{\aid}\rp}
\cfgrule{\num}{\cfgt{number}}
\cfgrule{\rcvp}{\lp\aid~$\ldots$\rp}
%\cfgrule{\npro}{\cfgt{number}}
\cfgrule{\aid}{\cfgt{ID}}
\cfgend
\end{minipage}
\end{lrbox}


\newsavebox{\boxEvalSyntax}
\begin{lrbox}{\boxEvalSyntax}
\begin{minipage}[c]{0.3\linewidth}
\cfgstart
\cfgrule{st}{\lp\cfgnt{ctp}\ \cfgnt{\epsnd}\ \cfgnt{\eprcv}\ \cfgnt{\epwait}\ \cfgnt{\epbarrier}\ \rcvp\rp}
\cfgrule{\epsnd}{\cfgt{\mt}\
   \cfgor\lp\cfgnt{\epsnd}~\lb\lp\cfgt{\num},\cfgt{\num}\rp~$\rightarrow$~\cfgt{\num}\rb\rp}
\cfgrule{\eprcv}{\cfgt{\mt}\
   \cfgor\lp\cfgnt{\eprcv}~\lb\lp\cfgt{\num},\cfgt{\num}\rp~$\rightarrow$~\cfgt{\num}\rb\rp}
\cfgrule{\epwait}{\cfgt{\mt}\
   \cfgor\lp\cfgnt{\epwait}~\lb\cfgnt{\aid}~$\rightarrow$~\cfgnt{rcv}\rb\rp}
\cfgrule{rcv}{\lp\lb\cfgt{\aid}~\cfgt{\num}~\cfgt{\num}\rb\ \ldots\rp}
\cfgrule{\epbarrier}{\cfgt{\mt}\
   \cfgor\lp\cfgnt{\epbarrier}~\lb\cfgnt{\aid}~$\rightarrow$~\cfgt{\num}\rb\rp}
\cfgend
\end{minipage}
\end{lrbox}

\begin{figure}
\begin{center}
\setlength{\tabcolsep}{15pt}
\begin{tabular}{cc}
\scalebox{0.75}{\usebox{\boxLangSyntax}}
&
\scalebox{0.75}{\usebox{\boxEvalSyntax}}
\\ \\
(a) & (b)
\end{tabular}
\end{center}
\caption{The language syntax with its evaluation syntax for the abstract machine in \figref{fig:machine} -- bold face indicates a terminal. (a) The input syntax. (b) The evaluation syntax.}
\label{fig:expr:stx}
\end{figure}

\examplefigthree


%%TODO: may move the machine state to where presenting the abstract machine.


